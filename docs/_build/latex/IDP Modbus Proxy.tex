%% Generated by Sphinx.
\def\sphinxdocclass{report}
\documentclass[letterpaper,10pt,english]{sphinxmanual}
\ifdefined\pdfpxdimen
   \let\sphinxpxdimen\pdfpxdimen\else\newdimen\sphinxpxdimen
\fi \sphinxpxdimen=.75bp\relax

\PassOptionsToPackage{warn}{textcomp}
\usepackage[utf8]{inputenc}
\ifdefined\DeclareUnicodeCharacter
 \ifdefined\DeclareUnicodeCharacterAsOptional
  \DeclareUnicodeCharacter{"00A0}{\nobreakspace}
  \DeclareUnicodeCharacter{"2500}{\sphinxunichar{2500}}
  \DeclareUnicodeCharacter{"2502}{\sphinxunichar{2502}}
  \DeclareUnicodeCharacter{"2514}{\sphinxunichar{2514}}
  \DeclareUnicodeCharacter{"251C}{\sphinxunichar{251C}}
  \DeclareUnicodeCharacter{"2572}{\textbackslash}
 \else
  \DeclareUnicodeCharacter{00A0}{\nobreakspace}
  \DeclareUnicodeCharacter{2500}{\sphinxunichar{2500}}
  \DeclareUnicodeCharacter{2502}{\sphinxunichar{2502}}
  \DeclareUnicodeCharacter{2514}{\sphinxunichar{2514}}
  \DeclareUnicodeCharacter{251C}{\sphinxunichar{251C}}
  \DeclareUnicodeCharacter{2572}{\textbackslash}
 \fi
\fi
\usepackage{cmap}
\usepackage[T1]{fontenc}
\usepackage{amsmath,amssymb,amstext}
\usepackage{babel}
\usepackage{times}
\usepackage[Bjarne]{fncychap}
\usepackage{sphinx}

\usepackage{geometry}

% Include hyperref last.
\usepackage{hyperref}
% Fix anchor placement for figures with captions.
\usepackage{hypcap}% it must be loaded after hyperref.
% Set up styles of URL: it should be placed after hyperref.
\urlstyle{same}

\addto\captionsenglish{\renewcommand{\figurename}{Fig.}}
\addto\captionsenglish{\renewcommand{\tablename}{Table}}
\addto\captionsenglish{\renewcommand{\literalblockname}{Listing}}

\addto\captionsenglish{\renewcommand{\literalblockcontinuedname}{continued from previous page}}
\addto\captionsenglish{\renewcommand{\literalblockcontinuesname}{continues on next page}}

\addto\extrasenglish{\def\pageautorefname{page}}





\title{IDP Modbus Proxy Documentation}
\date{Aug 22, 2018}
\release{1}
\author{Geoff Bruce-Payne, Yanosh Sabo}
\newcommand{\sphinxlogo}{\vbox{}}
\renewcommand{\releasename}{Release}
\makeindex

\begin{document}

\maketitle
\sphinxtableofcontents
\phantomsection\label{\detokenize{index::doc}}

\begin{savenotes}\sphinxattablestart
\centering
\begin{tabulary}{\linewidth}[t]{|T|T|T|}
\hline
\sphinxstartmulticolumn{2}%
\begin{varwidth}[t]{\sphinxcolwidth{2}{3}}
\raisebox{-0.5\height}{\sphinxincludegraphics[width=200\sphinxpxdimen,height=78\sphinxpxdimen]{{inmarsat}.jpg}}
\par
\vskip-\baselineskip\vbox{\hbox{\strut}}\end{varwidth}%
\sphinxstopmulticolumn
&
\raisebox{-0.5\height}{\sphinxincludegraphics[width=250\sphinxpxdimen,height=94\sphinxpxdimen]{{gse}.jpg}}
\\
\hline
\end{tabulary}
\par
\sphinxattableend\end{savenotes}


\begin{quote}\begin{description}
\item[{Date}] \leavevmode
2018-08-20

\item[{Revision}] \leavevmode
1

\item[{Authors}] \leavevmode\begin{itemize}
\item {} 
Geoff Bruce-Payne

\item {} 
Yanosh Sabo

\end{itemize}

\item[{Contributors}] \leavevmode\begin{itemize}
\item {} 
Robert Gauthier

\item {} 
Andrew Windridge

\end{itemize}

\end{description}\end{quote}


\chapter{Introduction}
\label{\detokenize{index:introduction}}
\sphinxhref{http://www.modbus.org}{Modbus} is a widely used industrial automation data protocol based on a Master/Slave architecture, often used in Supervisory Control and Data Acquisition (SCADA) systems.
Many industrial automation devices use Modbus protocol over a serial interface (RS-232 or RS-485).

IsatData Pro® is a \sphinxstyleemphasis{highly-reliable, global} satellite data service offered by \sphinxhref{https://www.inmarsat.com/service/isatdata-pro/}{Inmarsat}, well suited to transporting small amounts of machine-to-machine (M2M) and \sphinxstyleemphasis{Industrial Internet of Things} (IIoT) data to and from industrial devices (e.g. Modbus) located in remote locations.

The Modbus Proxy service is an \sphinxstyleemphasis{embedded} software application written in the \sphinxhref{https://www.lua.org}{Lua} programming language that runs on certain versions of IsatData Pro compatible satellite terminals manufactured by \sphinxhref{https://www.orbcomm.com/en/hardware/devices/st-series}{ORBCOMM}.

\begin{sphinxadmonition}{note}{Disclaimer}

While the information in this document has been prepared in good faith, no representation, warranty, assurance or undertaking (express or implied) is or will be made, and no responsibility or liability (howsoever arising) is or will be accepted by the Inmarsat group or any of its officers, employees or agents in relation to the adequacy, accuracy, completeness, reasonableness or fitness for purpose of the information in this document. All and any such responsibility and liability is expressly disclaimed and excluded to the maximum extent permitted by applicable law. INMARSAT is a trademark owned by the International Mobile Satellite Organisation, the Inmarsat LOGO is a trademark owned by Inmarsat (IP) Company Limited. Both trademarks are licensed to Inmarsat Global Limited. All other Inmarsat trade marks in this document are owned by Inmarsat Global Limited.
\end{sphinxadmonition}


\chapter{Contents}
\label{\detokenize{index:contents}}

\section{Overview}
\label{\detokenize{overview:overview}}\label{\detokenize{overview::doc}}

\subsection{Concept of Operation}
\label{\detokenize{overview:concept-of-operation}}
The IDP/ST satellite terminal is typically configured using a simple \sphinxstylestrong{config file} loaded via its main RS232 serial port prior to field deployment, and connects to a specific Modbus device (e.g. PLC, RTU, etc.) or network using either RS232 or RS485.  The config file is set up based on mapping the Modbus device manufacturer’s register definitions to relevant parameters for the application.

The satellite terminal polls the attached Modbus device(s) registers at a frequent interval such as every 10 seconds, and sends the relevant data over the satellite network at a longer interval such as once per hour.  Register data gets mapped to \sphinxstylestrong{parameters} and consolidated into an efficient single over-the-air \sphinxstylestrong{report} message for satellite transmission.  In this way, the proxy \sphinxstylestrong{report} is much more data cost efficient than polling the device over-the-air.  \sphinxstylestrong{Reports} can also be defined to \sphinxstyleemphasis{not} transmit periodically, instead being queried by the enterprise/cloud application as a more efficient data set than polling individual contiguous register blocks (some people refer to this kind of operation as a \sphinxstyleemphasis{Modbus shift}).

Additionally, analytics are configurable on register data that can detect relevant changes or trigger values and generate an immediate \sphinxstylestrong{alert} containing associated data from multiple registers.  This method avoids the risk of missing critical events in between scheduled data transmissions, while minimizing network costs by avoiding more frequent polling/reporting.

An enterprise/cloud application uses a secure (RESTful) web service API to retrieve the messages, at a much lower cost than actively polling native Modbus registers over the satellite link.  The enterprise application decodes the register data based on the Modbus device-specific data values, and processes those accordingly.

\begin{figure}[htbp]
\centering
\capstart

\noindent\sphinxincludegraphics[scale=1.0]{{conop}.jpg}
\caption{IDP Modbus Proxy Concept of Operation}\label{\detokenize{overview:id1}}\end{figure}


\subsection{Software Architecture}
\label{\detokenize{overview:software-architecture}}
The \sphinxcode{\sphinxupquote{Modbus}} service is an embedded software application that acts as a data proxy between an industrial device and a centralized/cloud application.  It is designed to run on ORBCOMM IsatData Pro (IDP and ST series) terminals using the Lua Services Framework (\sphinxstylestrong{LSF}) as a \sphinxstyleemphasis{user service} with a default service identifier number \sphinxstylestrong{SIN 200}.

The cloud application retrieves data from and sends data to the Modbus service using Over-The-Air message operations (\sphinxstylestrong{OTA API}).

Although the Modbus service is intended for standalone operation, it also exports adequate means for integration with other LSF services (\sphinxstylestrong{LSF API}) to provide software programmers a means to implement various extensions to the core feature set.
The Modbus service exports to the \sphinxstylestrong{LSF} the following:
\begin{itemize}
\item {} 
\sphinxcode{\sphinxupquote{From-Mobile Messages}} describing reports, alerts and response operations performed \sphinxstyleemphasis{OTA}

\item {} 
\sphinxcode{\sphinxupquote{To-Mobile Messages}} describing configuration and query operations performed \sphinxstyleemphasis{OTA}

\item {} 
\sphinxcode{\sphinxupquote{Properties}} containing configuration and data accessible remotely or via Lua operations

\item {} 
\sphinxcode{\sphinxupquote{Shell Commands}} operations that can be performed locally on the main serial port (if not in use for Modbus protocol)

\end{itemize}

\begin{sphinxadmonition}{note}{Note:}
The style of programming implementation for the various messages, events, and commands is case sensitive, and uses mixed case e.g. \sphinxcode{\sphinxupquote{baudRate}}.
\end{sphinxadmonition}

\begin{sphinxadmonition}{warning}{Warning:}
ORBCOMM IDP-series terminals (e.g. IDP-680, IDP-782) use a different version of Lua than ST-series terminals (e.g. ST6100).  The Modbus service has been designed to be compatible with both terminal hardware platforms, but requires the use of different package builds for each family type.  The service name is \sphinxcode{\sphinxupquote{Modbus}} in both cases, but the package file uses a suffix either \textendash{}IDP or \textendash{}ST to indicate the target hardware platform.
\end{sphinxadmonition}

\begin{sphinxadmonition}{note}{Note:}
Software programmers wishing to customize the service using their own Lua code, should consult with the relevant ORBCOMM product documentation for the target hardware platform (IDP series or ST series).
\end{sphinxadmonition}


\section{Functional Overview}
\label{\detokenize{funcspec:functional-overview}}\label{\detokenize{funcspec::doc}}

\subsection{Operation}
\label{\detokenize{funcspec:operation}}

\subsubsection{Modbus Device Communication}
\label{\detokenize{funcspec:modbus-device-communication}}
A Modbus \sphinxstylestrong{device} (e.g. industrial automation controller, PLC, RTU, meter, etc.) is connected to a serial \sphinxstylestrong{port} on the satellite terminal.

Relevant Modbus register values are mapped to \sphinxstylestrong{parameters} that can be used to generate periodic \sphinxstylestrong{reports} and real-time \sphinxstylestrong{alerts}.

Each \sphinxstylestrong{report} can include multiple data types and values, optimized into a small satellite message.  This delivers a much lower data cost compared to polling native Modbus registers over-the-air.

Each \sphinxstylestrong{alert} is set up to trigger on a \sphinxstylestrong{parameter} threshold value, either high/low or a change from the previously reported value.  Alerts can include additional parameter data in addition to the trigger parameter value.

\sphinxstylestrong{Parameters} can be queried remotely in addition to periodic reports and alerts, either individually or in groups.  This method is more efficient than polling native Modbus registers which require separate polls for discontiguous register ranges and/or different data types.

Configuration of \sphinxstylestrong{ports}, \sphinxstylestrong{devices}, \sphinxstylestrong{parameters}, \sphinxstylestrong{reports} and \sphinxstylestrong{alerts} is normally done by creating a \sphinxstyleemphasis{configuration file} called \sphinxcode{\sphinxupquote{config.dat}} that is included in a build package to load on the satellite terminal prior to deployment in the field.

An \sphinxstylestrong{OTA API} provides satellite \sphinxstylestrong{message} constructs for remote configuration, query of parameters, and direct read/write of Modbus registers.
These \sphinxstylestrong{message} structures are sent using Inmarsat’s \sphinxhref{https://developer.inmarsat.com/content/isatdatapro-messaging-api}{IDP Messaging API}.


\paragraph{Periodic Reports}
\label{\detokenize{funcspec:periodic-reports}}
By default, each \sphinxstylestrong{report} is sent immediately when configured or after a terminal reset.  Subsequent reports are sent at the defined interval in \sphinxstyleemphasis{seconds} relative to the previous report.


\paragraph{Time of Day Reports}
\label{\detokenize{funcspec:time-of-day-reports}}
A report can be \sphinxstyleemphasis{optionally} configured to send data from a specific time of day.
Because having many devices send data at the exact same time causes problems on the satellite network, when using Time of Day reports you must configure a \sphinxstylestrong{window} to distribute the time of day measurements on the satellite network.
In this case, the measurement is taken as a \sphinxstyleemphasis{snapshot} at the specific time of day, and then sent a short while afterward.

Time of Day reports from a given terminal will be sent at the same offset time within the window, each day.  However 2 terminals configured with same time of day and window, may transmit at different times relative to the other.


\paragraph{Error Handling}
\label{\detokenize{funcspec:error-handling}}
Modbus and serial communications errors are captured and sent as {\hyperref[\detokenize{otaapi:error}]{\sphinxcrossref{\DUrole{std,std-ref}{alerts}}}}, as well as logged for diagnostic retrieval.  The following error scenarios are handled:
\begin{itemize}
\item {} 
Successive read timeouts are trapped as \sphinxstyleemphasis{loss of communication} based on the configured Timeouts (Rx, Tx) after the configured number of retries.
Communication is assumed to be re-established after the next successful read.

\item {} 
Modbus protocol errors

\end{itemize}


\begin{savenotes}\sphinxattablestart
\centering
\sphinxcapstartof{table}
\sphinxcaption{Diagnostics Messages}\label{\detokenize{funcspec:id2}}\label{\detokenize{funcspec:diagnostics-messages}}
\sphinxaftercaption
\begin{tabulary}{\linewidth}[t]{|T|T|}
\hline
\sphinxstyletheadfamily 
Error Code
&\sphinxstyletheadfamily 
Error Description
\\
\hline
0
&
OK. Communication with Modbus device restored.
\\
\hline
1
&
Communication with Modbus device lost.
\\
\hline
2
&
Reporting configuration missing or reportingId not found.
\\
\hline
3
&
Analytics configuration missing or alertId not found.
\\
\hline
4
&
Failed to acknowledge geolocation alert.
\\
\hline
\end{tabulary}
\par
\sphinxattableend\end{savenotes}


\subsubsection{Geolocation and Movement Alerts}
\label{\detokenize{funcspec:geolocation-and-movement-alerts}}
Geolocation is an \sphinxstyleemphasis{optional} feature that uses a reference location and periodically checks the terminal’s location using internal GNSS to determine if the terminal has moved more than a certain distance in a certain time.
The \sphinxstylestrong{reference location} is determined by either:
\begin{itemize}
\item {} 
the current location when geolocation alerts are configured

\item {} 
remote configuration using the \sphinxstylestrong{OTA API}

\end{itemize}

To stop geolocation alerts from repeatedly sending once the terminal has moved, you must send an \sphinxstylestrong{acknowledgement of movement} to the terminal using the \sphinxstylestrong{OTA API}.

\begin{sphinxadmonition}{note}{Note:}
GNSS has inherent error sources, so the geolocation feature takes multiple samples that must exceed the distance for a configured amount of time before a location change alert is sent.
\end{sphinxadmonition}


\subsubsection{Low Power Operation}
\label{\detokenize{funcspec:low-power-operation}}
Power saving is an \sphinxstyleemphasis{optional} feature that changes the behaviour of the satellite terminal to conserve power, trading off against certain responsiveness to changes.

The following table summarizes the differences between normal and low power mode:


\begin{savenotes}\sphinxattablestart
\centering
\sphinxcapstartof{table}
\sphinxcaption{Low Power Mode Behaviour}\label{\detokenize{funcspec:id3}}\label{\detokenize{funcspec:lpm-behaviour}}
\sphinxaftercaption
\begin{tabular}[t]{|*{4}{\X{1}{4}|}}
\hline
\sphinxstyletheadfamily 
Terminal Activity
&\sphinxstyletheadfamily 
Default Behaviour
&\sphinxstyletheadfamily 
LPM Behaviour
&\sphinxstyletheadfamily 
LSF API Property Interaction {[}SIN:PIN{]}
\\
\hline
Incoming Messages
&
Immediate
&
Check every 60 seconds
&
\begin{DUlineblock}{0em}
\item[] Exit LPM {[}27:11{]}=0
\item[] Enter LPM {[}27:11{]}=6
\end{DUlineblock}
\\
\hline
Geolocation checks
&
Every 60 seconds
&
Every 360 seconds
&
\begin{DUlineblock}{0em}
\item[] Exit LPM {[}20:15{]}=2
\item[] Enter LPM {[}20:15{]}=60
\end{DUlineblock}
\\
\hline
Processing
&
Realtime
&
Delayed
&
\begin{DUlineblock}{0em}
\item[] Exit LPM {[}16:4{]}=0
\item[] Enter LPM {[}16:4{]}=1
\end{DUlineblock}
\\
\hline
LED indicator
&
Flashes every 4 minutes
&
No flash
&
\begin{DUlineblock}{0em}
\item[] Exit LPM {[}16:6{]}=0; {[}16:7{]}=4
\item[] Enter LPM {[}16:6{]}=1; {[}16:7{]}=0
\end{DUlineblock}
\\
\hline
RS232
&
Always on
&
Switch off when no data
&
\begin{DUlineblock}{0em}
\item[] Exit LPM {[}22:6{]}=1; {[}22:15{]}=2; {[}22:43{]}=1; {[}22:44{]}=2
\item[] Enter LPM {[}22:6{]}=2; {[}22:15{]}=0; {[}22:43{]}=2; {[}22:44{]}=0
\end{DUlineblock}
\\
\hline
Satellite Blockage Handling
&
Always reacquire (Mobile, Powered)
&
Attempt every hour (Mobile, Parked)
&
\begin{DUlineblock}{0em}
\item[] Exit LPM {[}27:10{]}=1
\item[] Enter LPM {[}27:10{]}=5
\end{DUlineblock}
\\
\hline
\end{tabular}
\par
\sphinxattableend\end{savenotes}


\subsubsection{Metadata}
\label{\detokenize{funcspec:metadata}}
The following \sphinxstyleemphasis{metadata} is optionally available for each \sphinxstylestrong{report} and \sphinxstylestrong{alert}, which can use the default settings or override on a per report/alert basis:
\begin{itemize}
\item {} 
\sphinxcode{\sphinxupquote{latitude}} is the latitude component of geolocation, expressed in 1/1000th minutes.  \sphinxstylestrong{Divide by 60,000} to get \sphinxstyleemphasis{decimal degrees} up to 5 decimal places precision.

\item {} 
\sphinxcode{\sphinxupquote{longitude}} is the longitude component of geolocation, expressed in 1/1000th minutes.  \sphinxstylestrong{Divide by 60,000} to get \sphinxstyleemphasis{decimal degrees} up to 5 decimal places precision.

\item {} 
\sphinxcode{\sphinxupquote{speed}} is the speed component of geolocation, expressed in \sphinxstyleemphasis{knots}.

\item {} 
\sphinxcode{\sphinxupquote{altitude}} is the altitude component of geolocation, expressed in \sphinxstyleemphasis{meters}.

\item {} 
\sphinxcode{\sphinxupquote{distance}} is the distance traveled, in meters, in between 2 geolocation reports.

\item {} 
\sphinxcode{\sphinxupquote{timestamp}} is the time that the \sphinxstylestrong{report} or \sphinxstylestrong{alert} was triggered, in seconds since \sphinxstyleemphasis{1970-01-01T00:00:00Z}.

\item {} 
\sphinxcode{\sphinxupquote{paramTimestamp}} is the time that an individual \sphinxstylestrong{parameter} was read from the Modbus device, in seconds since \sphinxstyleemphasis{1970-01-01T00:00:00Z}.

\end{itemize}


\subsection{Feature Summary}
\label{\detokenize{funcspec:feature-summary}}
The Modbus proxy service has been designed based on the following requirements:


\subsubsection{General}
\label{\detokenize{funcspec:general}}\begin{enumerate}
\item {} 
The following hardware variants are supported:
\begin{itemize}
\item {} 
ST6100 (satellite-only)

\item {} 
IDP-680/690 (satellite-only)

\item {} 
IDP-782 (cellular/satellite)

\item {} 
IDP-800 (self-powered satellite-only)

\end{itemize}

\item {} 
The Modbus \sphinxstyleemphasis{user service} is completely self-contained in terms of documentation and examples for an integrator, without dependencies on non-core \sphinxstylestrong{LSF} services.
A typical user does not need to be intimately familiar with the ORBCOMM product documentation and Lua Service Framework.

\end{enumerate}


\subsubsection{Physical Communication Layer}
\label{\detokenize{funcspec:physical-communication-layer}}\begin{enumerate}
\item {} 
Configurable to work on any of the supported terminal’s RS232 or RS485 interface(s).

\item {} 
Supports serial baud rates on industry-standard settings from 1200 to 115200, as well as it supports configuration of parity, data bits and stop bits.

\item {} 
If communication with a slave device is lost, the service sends a corresponding alert.

\end{enumerate}


\subsubsection{Data Communication (Modbus Protocol) Layer}
\label{\detokenize{funcspec:data-communication-modbus-protocol-layer}}\begin{enumerate}
\item {} 
Supports operation on one or multiple terminal serial ports (one Modbus network per serial port).

\item {} 
Configurable for Modbus RTU or ASCII protocol variants (one variant per Modbus network).

\item {} 
Configurable Modbus slave response timeout, with a default value set to 3 seconds.

\item {} 
Supports multiple Modbus device (slave) addresses when using RS485 port on a multi-drop Modbus network.

\item {} 
Configurable for 0-based (default, native) or 1-based (PLC) Modbus addressing per each slave device.

\item {} 
The following types of Modbus addresses supported:
\begin{itemize}
\item {} 
Coils

\item {} 
Discrete inputs

\item {} 
Analog inputs

\item {} 
Holding (aka analog input/output) register

\end{itemize}

\item {} 
Handles Modbus protocol errors by logging to the log core service. Logs can be retrieved remotely utilizing corresponding log service messages.

\end{enumerate}


\subsubsection{Data Interpretation}
\label{\detokenize{funcspec:data-interpretation}}\begin{enumerate}
\item {} 
Maps Modbus register values to parameters using the following configuration options per parameter:
\begin{itemize}
\item {} 
\sphinxcode{\sphinxupquote{encoding}} data type (signed/unsigned integer or 8/16/32 bits, Boolean, 32-bit floating point, ASCII)

\item {} 
\sphinxcode{\sphinxupquote{length}} data size (number of contiguous registers)

\end{itemize}

\item {} 
Configurable byte order and word order (Endianness) on a per device basis).

\item {} 
Data sampling (Modbus polling) interval is configurable value on a per-device basis.

\item {} 
If specified in the configuration for the specific parameter, the service generates alerts if a measurement value exceeds some threshold either high, low, or a change of X amount.

\end{enumerate}


\subsubsection{Data Reporting and Register Read/Write}
\label{\detokenize{funcspec:data-reporting-and-register-read-write}}\begin{enumerate}
\item {} 
OTA reports are configurable to contain multiple (up to 100) different register data values of same or different type, as individual parameter data fields within a single message.

\item {} 
Reports can be sent at a configurable interval.

\item {} 
Time-of-day based register reads can be captured as a “snapshot” into a timestamped report with data delivery distributed over a small window to load balance the satellite network.

\item {} 
Supports native/transparent Modbus read/write commands and responses OTA, on-demand, using a messaging API.

\item {} 
Effectively allows “remapping registers”, since parameter data fields do not need to use contiguous register values.

\item {} 
Timestamps can optionally be included in reports or alerts on a per-parameter basis. Timestamp (UNIX epoch format) shall be that of the most recent Modbus slave response prior to message transmission.

\item {} 
Location can optionally be included in reports and alerts.  Location resolution is approximately +/- 1m (1/1000 minutes, recognizing that GPS accuracy is typically 5-10m), +/- 1 km/h, +/- 1-degree heading, and +/- 1m altitude

\end{enumerate}


\subsubsection{Alerts and Data Analytics}
\label{\detokenize{funcspec:alerts-and-data-analytics}}\begin{enumerate}
\item {} 
Supports configuration of up to 245 unique alerts, triggered by interpreting specified parameter data values including:
\begin{itemize}
\item {} 
High value threshold

\item {} 
Low value threshold

\item {} 
Return-to-normal sends an alert when the high or low value last reported exceeded threshold but returned to normal range

\item {} 
Change value relative to last read value

\end{itemize}

\end{enumerate}


\subsubsection{Geolocation}
\label{\detokenize{funcspec:geolocation}}\begin{enumerate}
\item {} 
Optional location change alert based on configurable distance change relative to last reported location (persisting in time to prevent random GNSS errors triggering false alerts).

\end{enumerate}


\subsubsection{Low Power Operation}
\label{\detokenize{funcspec:id1}}\begin{enumerate}
\item {} 
The service supports optional power saving to support battery-powered installations.

\end{enumerate}


\subsubsection{Service Configuration}
\label{\detokenize{funcspec:service-configuration}}\begin{enumerate}
\item {} 
The service contains no default polling, reporting or alert configurations.  It must be explicitly configured.

\item {} 
The primary method of configuration is pre-configuration using a text file, loaded on the satellite terminal using its main console RS232 port.

\item {} 
Supports manual local configuration using console commands on the main RS232 interface.  Console commands present a similar structure as the configuration text file.

\item {} 
Supports over-the-air (OTA) configuration query and changes using a messaging API.

\item {} 
Supports embedded software local configuration using the Lua Service Framework API.

\item {} 
All configurations are stored in non-volatile memory, and persist satellite terminal reset.

\item {} 
Optional configuration of power-saving mode to minimize energy consumption.

\end{enumerate}


\subsubsection{Diagnostics}
\label{\detokenize{funcspec:diagnostics}}\begin{enumerate}
\item {} 
Collects Modbus diagnostics data such as timeouts and errors, made available by polling debug and error logs remotely using the messaging API.

\end{enumerate}


\subsubsection{Lua Service Framework API}
\label{\detokenize{funcspec:lua-service-framework-api}}\begin{enumerate}
\item {} 
The service provides an effective interface for integration with other services and Agents via properties, events, function calls and messages

\end{enumerate}


\section{How to Use}
\label{\detokenize{usage:how-to-use}}\label{\detokenize{usage::doc}}\begin{enumerate}
\item {} 
Obtain manufacturer documentation for your Modbus device indicating {\hyperref[\detokenize{usage:modbus-register-definitions}]{\sphinxcrossref{\DUrole{std,std-ref}{register definitions}}}}.

\item {} 
Determine whether the device will be using RS-232 or RS-485 communication, and the settings for \sphinxcode{\sphinxupquote{baudRate}} that will be used.

\item {} 
Determine which register data \sphinxstylestrong{parameters} are relevant for your application requirements, and how those are presented.

\item {} 
Decide which register data are to be captured in periodic \sphinxstylestrong{reports}, and at what \sphinxstyleemphasis{interval} that set of data is most useful.

\item {} 
Decide which register data are \sphinxstyleemphasis{time-critical} and need to be captured in real-time with an \sphinxstylestrong{alert}.

\item {} 
Create and save your \sphinxcode{\sphinxupquote{config.dat}} \sphinxstylestrong{config file} using a text editor based on the {\hyperref[\detokenize{usage:config-file-template}]{\sphinxcrossref{\DUrole{std,std-ref}{template}}}}.

\item {} 
Create your build package using {\hyperref[\detokenize{usage:solution-studio-package}]{\sphinxcrossref{\DUrole{std,std-ref}{Solution Studio}}}}.

\end{enumerate}


\subsection{Example Use}
\label{\detokenize{usage:example-use}}

\subsubsection{Create Your Configuration File}
\label{\detokenize{usage:create-your-configuration-file}}
In this example we will be reading data from a weather station.

Manufacturer documentation: \sphinxhref{https://www.lufft.com/products/compact-weather-sensors-293/ws504-umb-smart-weather-sensor-1836/}{Lufft Compact Weather Sensor WS504}

The WS504 requires some manual configuration to use RS-485 in Modbus mode, using 9600-8-N-1 serial.  The default Modbus Slave address is 1, which will be our \sphinxcode{\sphinxupquote{networkId=1}}.

We are only connecting to one device, so we will set \sphinxcode{\sphinxupquote{deviceId=1}}.

We find the registers of interest, which are all read-only values (\sphinxcode{\sphinxupquote{registerType=analog}}) and all 16-bit signed integers (\sphinxcode{\sphinxupquote{encoding=int16}}).
The manufacturer document provides both the PLC base-1 address (\sphinxstyleemphasis{register number}) and the native \sphinxstyleemphasis{register address}, so we can use the latter as \sphinxcode{\sphinxupquote{address}} and
leave the default configuration \sphinxcode{\sphinxupquote{plcBaseAddress=0}}.

\phantomsection\label{\detokenize{usage:modbus-register-definitions}}
And we map them to \sphinxstylestrong{parameters} using \sphinxcode{\sphinxupquote{paramId}}, \sphinxcode{\sphinxupquote{registerType}}, \sphinxcode{\sphinxupquote{address}} and \sphinxcode{\sphinxupquote{encoding}}:


\begin{savenotes}\sphinxattablestart
\centering
\sphinxcapstartof{table}
\sphinxcaption{Configuration File Template}\label{\detokenize{usage:id1}}\label{\detokenize{usage:config-file-template}}
\sphinxaftercaption
\begin{tabulary}{\linewidth}[t]{|T|T|T|T|T|T|T|T|T|T|T|T|}
\hline
\sphinxstyletheadfamily 
Data/Sensor
&\sphinxstyletheadfamily 
Register Type
&\sphinxstyletheadfamily 
Register Number
&\sphinxstyletheadfamily 
Register Address
&\sphinxstyletheadfamily 
Units
&\sphinxstyletheadfamily 
Decoding
&\sphinxstyletheadfamily 
Min
&\sphinxstyletheadfamily 
Max
&\sphinxstyletheadfamily 
paramId
&\sphinxstyletheadfamily 
registerType
&\sphinxstyletheadfamily 
address
&\sphinxstyletheadfamily 
encoding
\\
\hline
Air temperature
&
Read-only (Input Register)
&
32
&
31
&
Celsius
&
Factor 10, signed
&
-500
&
600
&
1
&
analog
&
31
&
int16
\\
\hline
Relative humidity
&
Read-only (Input Register)
&
11
&
10
&
\%RH
&
Factor 10, signed
&
0
&
1000
&
2
&
analog
&
10
&
int16
\\
\hline
Relative air pressure
&
Read-only (Input Register)
&
15
&
14
&
hPa
&
Factor 10, signed
&
3000
&
12000
&
3
&
analog
&
14
&
int16
\\
\hline
Wind speed, average
&
Read-only (Input Register)
&
87
&
86
&
km/h
&
Factor 10, signed
&
0
&
2700
&
4
&
analog
&
86
&
int16
\\
\hline
Wind direction
&
Read-only (Input Register)
&
19
&
18
&
degrees
&
Factor 10, signed
&
0
&
3599
&
5
&
analog
&
18
&
int16
\\
\hline
\end{tabulary}
\par
\sphinxattableend\end{savenotes}

We want to send:
\begin{itemize}
\item {} 
All our parameters as a \sphinxstylestrong{report} (\sphinxcode{\sphinxupquote{reportingId=1}}) every hour (\sphinxcode{\sphinxupquote{interval=3600}})

\item {} 
Also send an immediate \sphinxstylestrong{alert} (\sphinxcode{\sphinxupquote{alertId=1}}) if the temperature (\sphinxcode{\sphinxupquote{paramId=1}}) goes above 35 degrees Celsius and another alert when temperature returns below 30 Celsius.
The manufacturer document tells us the Modbus register contains temperature in degrees Celsius with a \sphinxstyleemphasis{factor 10}, so our thresholds will be \sphinxcode{\sphinxupquote{highON=350}} and \sphinxcode{\sphinxupquote{highOFF=300}}

\end{itemize}

In order to get real-time temperature alerts, we will set \sphinxcode{\sphinxupquote{pollInterval=10}} seconds.

\phantomsection\label{\detokenize{usage:example-config-dat}}
Our resulting \sphinxcode{\sphinxupquote{config.dat}} file looks like:

\fvset{hllines={, ,}}%
\begin{sphinxVerbatim}[commandchars=\\\{\}]
\PYG{o}{/}\PYG{o}{*} \PYG{n}{Example} \PYG{n}{Modbus} \PYG{n}{Proxy} \PYG{n}{configuration} \PYG{k}{for} \PYG{n}{Lufft} \PYG{n}{WS504} \PYG{o}{*}\PYG{o}{/}
\PYG{o}{/}\PYG{o}{*} \PYG{n}{PORTS} \PYG{o}{*}\PYG{o}{/}
\PYG{n}{port}\PYG{o}{=}\PYG{n}{rs485}\PYG{p}{;}\PYG{n}{baudRate}\PYG{o}{=}\PYG{l+m+mi}{9600}\PYG{p}{;}\PYG{n}{parity}\PYG{o}{=}\PYG{n}{none}\PYG{p}{;}\PYG{n}{mode}\PYG{o}{=}\PYG{n}{rtu}
\PYG{o}{/}\PYG{o}{*} \PYG{n}{DEVICES} \PYG{o}{*}\PYG{o}{/}
\PYG{n}{deviceId}\PYG{o}{=}\PYG{l+m+mi}{1}\PYG{p}{;}\PYG{n}{networkId}\PYG{o}{=}\PYG{l+m+mi}{1}\PYG{p}{;}\PYG{n}{port}\PYG{o}{=}\PYG{n}{rs485}\PYG{p}{;}\PYG{n}{pollInterval}\PYG{o}{=}\PYG{l+m+mi}{10}\PYG{p}{;}\PYG{n}{byteOrder}\PYG{o}{=}\PYG{n}{msb}\PYG{p}{;}\PYG{n}{wordOrder}\PYG{o}{=}\PYG{n}{msw}\PYG{p}{;}\PYG{n}{serialRxTimeout}\PYG{o}{=}\PYG{l+m+mi}{1000}\PYG{p}{;}\PYG{n}{serialTxTimeout}\PYG{o}{=}\PYG{l+m+mi}{1000}\PYG{p}{;}\PYG{n}{plcBaseAddress}\PYG{o}{=}\PYG{l+m+mi}{0}\PYG{p}{;}\PYG{n}{retries}\PYG{o}{=}\PYG{l+m+mi}{3}
\PYG{o}{/}\PYG{o}{*} \PYG{n}{PARAMETERS} \PYG{o}{*}\PYG{o}{/}
\PYG{o}{/}\PYG{o}{*} \PYG{n}{Temperature} \PYG{o+ow}{in} \PYG{n}{read}\PYG{o}{\PYGZhy{}}\PYG{n}{only} \PYG{n}{register} \PYG{l+m+mi}{32} \PYG{o}{*}\PYG{o}{/}
\PYG{n}{paramId}\PYG{o}{=}\PYG{l+m+mi}{1}\PYG{p}{;}\PYG{n}{deviceId}\PYG{o}{=}\PYG{l+m+mi}{1}\PYG{p}{;}\PYG{n}{registerType}\PYG{o}{=}\PYG{n}{analog}\PYG{p}{;}\PYG{n}{address}\PYG{o}{=}\PYG{l+m+mi}{31}\PYG{p}{;}\PYG{n}{encoding}\PYG{o}{=}\PYG{n}{int16}
\PYG{o}{/}\PYG{o}{*} \PYG{n}{Humidity} \PYG{o+ow}{in} \PYG{n}{read}\PYG{o}{\PYGZhy{}}\PYG{n}{only} \PYG{n}{register} \PYG{l+m+mi}{11} \PYG{o}{*}\PYG{o}{/}
\PYG{n}{paramId}\PYG{o}{=}\PYG{l+m+mi}{2}\PYG{p}{;}\PYG{n}{deviceId}\PYG{o}{=}\PYG{l+m+mi}{1}\PYG{p}{;}\PYG{n}{registerType}\PYG{o}{=}\PYG{n}{analog}\PYG{p}{;}\PYG{n}{address}\PYG{o}{=}\PYG{l+m+mi}{10}\PYG{p}{;}\PYG{n}{encoding}\PYG{o}{=}\PYG{n}{int16}
\PYG{o}{/}\PYG{o}{*} \PYG{n}{Air} \PYG{n}{Pressure} \PYG{o+ow}{in} \PYG{n}{read}\PYG{o}{\PYGZhy{}}\PYG{n}{only} \PYG{n}{register} \PYG{l+m+mi}{15} \PYG{o}{*}\PYG{o}{/}
\PYG{n}{paramId}\PYG{o}{=}\PYG{l+m+mi}{3}\PYG{p}{;}\PYG{n}{deviceId}\PYG{o}{=}\PYG{l+m+mi}{1}\PYG{p}{;}\PYG{n}{registerType}\PYG{o}{=}\PYG{n}{analog}\PYG{p}{;}\PYG{n}{address}\PYG{o}{=}\PYG{l+m+mi}{14}\PYG{p}{;}\PYG{n}{encoding}\PYG{o}{=}\PYG{n}{int16}
\PYG{o}{/}\PYG{o}{*} \PYG{n}{Wind} \PYG{n}{speed} \PYG{o+ow}{in} \PYG{n}{read}\PYG{o}{\PYGZhy{}}\PYG{n}{only} \PYG{n}{register} \PYG{l+m+mi}{87} \PYG{o}{*}\PYG{o}{/}
\PYG{n}{paramId}\PYG{o}{=}\PYG{l+m+mi}{4}\PYG{p}{;}\PYG{n}{deviceId}\PYG{o}{=}\PYG{l+m+mi}{1}\PYG{p}{;}\PYG{n}{registerType}\PYG{o}{=}\PYG{n}{analog}\PYG{p}{;}\PYG{n}{address}\PYG{o}{=}\PYG{l+m+mi}{86}\PYG{p}{;}\PYG{n}{encoding}\PYG{o}{=}\PYG{n}{int16}
\PYG{o}{/}\PYG{o}{*} \PYG{n}{Wind} \PYG{n}{direction} \PYG{o+ow}{in} \PYG{n}{read}\PYG{o}{\PYGZhy{}}\PYG{n}{only} \PYG{n}{register} \PYG{l+m+mi}{19} \PYG{o}{*}\PYG{o}{/}
\PYG{n}{paramId}\PYG{o}{=}\PYG{l+m+mi}{5}\PYG{p}{;}\PYG{n}{deviceId}\PYG{o}{=}\PYG{l+m+mi}{1}\PYG{p}{;}\PYG{n}{registerType}\PYG{o}{=}\PYG{n}{analog}\PYG{p}{;}\PYG{n}{address}\PYG{o}{=}\PYG{l+m+mi}{18}\PYG{p}{;}\PYG{n}{encoding}\PYG{o}{=}\PYG{n}{int16}
\PYG{o}{/}\PYG{o}{*} \PYG{n}{REPORTS} \PYG{o}{*}\PYG{o}{/}
\PYG{n}{reportingId}\PYG{o}{=}\PYG{l+m+mi}{1}\PYG{p}{;}\PYG{n}{interval}\PYG{o}{=}\PYG{l+m+mi}{3600}\PYG{p}{;}\PYG{n}{paramIds}\PYG{o}{=}\PYG{l+m+mi}{1}\PYG{p}{,}\PYG{l+m+mi}{2}\PYG{p}{,}\PYG{l+m+mi}{3}\PYG{p}{,}\PYG{l+m+mi}{4}\PYG{p}{,}\PYG{l+m+mi}{5}
\PYG{o}{/}\PYG{o}{*} \PYG{n}{ALERTS} \PYG{o}{*}\PYG{o}{/}
\PYG{n}{alertId}\PYG{o}{=}\PYG{l+m+mi}{1}\PYG{p}{;}\PYG{n}{paramId}\PYG{o}{=}\PYG{l+m+mi}{1}\PYG{p}{;}\PYG{n}{maxON}\PYG{o}{=}\PYG{l+m+mi}{350}\PYG{p}{;}\PYG{n}{maxOFF}\PYG{o}{=}\PYG{l+m+mi}{300}\PYG{p}{;}\PYG{n}{paramIds}\PYG{o}{=}\PYG{l+m+mi}{1}\PYG{p}{,}\PYG{l+m+mi}{2}\PYG{p}{,}\PYG{l+m+mi}{3}\PYG{p}{,}\PYG{l+m+mi}{4}\PYG{p}{,}\PYG{l+m+mi}{5}
\end{sphinxVerbatim}


\subsubsection{Create Your Build Package}
\label{\detokenize{usage:create-your-build-package}}\label{\detokenize{usage:solution-studio-package}}
\begin{sphinxadmonition}{note}{Note:}
Use of the tools described below requires that you have a PC with a modern version of Windows OS.
\end{sphinxadmonition}
\begin{enumerate}
\item {} 
Download the latest \sphinxstylestrong{Modbus Proxy} package file from \sphinxstyleemphasis{TBD} or contact your \sphinxstyleemphasis{Inmarsat Representative}.  Make sure you select the correct build for \sphinxstylestrong{IDP} or \sphinxstylestrong{ST} terminals.

\item {} 
Download the latest terminal firmware package for your \sphinxstylestrong{terminal model} from \sphinxhref{http://support.skywave.com/Support\_AllDownload.aspx}{ORBCOMM} or contact your \sphinxstyleemphasis{Inmarsat Representative}.

\item {} 
Download and install the latest \sphinxstylestrong{Developer Toolkit} from ORBCOMM or contact your \sphinxstyleemphasis{Inmarsat Representative}.

\item {} 
Open \sphinxstylestrong{Solution Studio} from the \sphinxstylestrong{Developer Toolkit}
\begin{quote}
\begin{enumerate}
\item {} 
Select \sphinxmenuselection{Add}, browse to the firmware package file and add it.

\item {} 
Select \sphinxmenuselection{Add}, browse to the Modbus Proxy package file and add it.

\item {} 
Select \sphinxcode{\sphinxupquote{Modbus}} in the \sphinxstyleemphasis{Services} pane, then click \sphinxmenuselection{Properties}

\end{enumerate}
\phantomsection\label{\detokenize{usage:changedefaultmsgbitmap}}\begin{enumerate}
\item {} 
Select \sphinxmenuselection{Add} then select \sphinxcode{\sphinxupquote{004 - defaultMsgBitMap}} and set the value to \sphinxcode{\sphinxupquote{04}} to enable \sphinxcode{\sphinxupquote{timestamp}} on \sphinxstylestrong{reports} and \sphinxstylestrong{alerts}

\item {} 
Return to the main GUI window by clicking \sphinxmenuselection{OK} when prompted.

\item {} 
Click \sphinxmenuselection{Edit}, then click \sphinxmenuselection{Add} and browse to your \sphinxcode{\sphinxupquote{config.dat}} file.  Click \sphinxmenuselection{OK} until you return to the main window.

\item {} 
Select \sphinxmenuselection{Edit \(\rightarrow\) Core Service Properties} then set the following similar to \sphinxcode{\sphinxupquote{defaultMsgBitMap}} above:
\begin{itemize}
\item {} 
\sphinxmenuselection{Service \(\rightarrow\) 016-system} \textendash{}\textgreater{} \sphinxcode{\sphinxupquote{001 - executionWatchdogTimeout}} ==\textgreater{} \sphinxcode{\sphinxupquote{30000}}

\item {} 
\sphinxmenuselection{Service \(\rightarrow\) 016-system} \textendash{}\textgreater{} \sphinxcode{\sphinxupquote{003 - autoGCMemThreshold}} ==\textgreater{} \sphinxcode{\sphinxupquote{5}}

\item {} 
\sphinxmenuselection{Service \(\rightarrow\) 023-log} \textendash{}\textgreater{} \sphinxcode{\sphinxupquote{031 - traceSuppress3}} ==\textgreater{} \sphinxcode{\sphinxupquote{1B 14}}

\end{itemize}

\item {} 
Click \sphinxmenuselection{OK} until you return to the main window.

\item {} 
Select \sphinxmenuselection{Edit \(\rightarrow\) Post-Load Commands} and type the following, depending on your terminal model:
\begin{quote}


\begin{savenotes}\sphinxattablestart
\centering
\sphinxcapstartof{table}
\sphinxcaption{Post-load Commands (when using RS485 for Modbus)}\label{\detokenize{usage:id2}}
\sphinxaftercaption
\begin{tabular}[t]{|*{2}{\X{1}{2}|}}
\hline

\sphinxstylestrong{IDP series}
&
\sphinxstylestrong{ST series}
\\
\hline
\begin{DUlineblock}{0em}
\item[] \sphinxcode{\sphinxupquote{trace channel main}}
\item[] \sphinxcode{\sphinxupquote{trace baud 9600}}
\item[] \sphinxcode{\sphinxupquote{trace enable}}
\end{DUlineblock}
&
\begin{DUlineblock}{0em}
\item[] \sphinxcode{\sphinxupquote{log channel main}}
\item[] \sphinxcode{\sphinxupquote{log baud 9600}}
\item[] \sphinxcode{\sphinxupquote{log serial enable}}
\end{DUlineblock}
\\
\hline
\end{tabular}
\par
\sphinxattableend\end{savenotes}
\end{quote}

\item {} 
Select \sphinxmenuselection{File \(\rightarrow\) Export Package} and save the resulting build file e.g. \sphinxcode{\sphinxupquote{my\_modbus.idppkg}}.

\item {} 
Select \sphinxmenuselection{File \(\rightarrow\) Export Message Definitions} and save the resulting \sphinxcode{\sphinxupquote{my\_modbus.idpmsg}} file.

\item {} 
Save your Solution Studio file e.g. \sphinxcode{\sphinxupquote{my\_modbus.idpsln}}.

\end{enumerate}

\begin{sphinxadmonition}{tip}{Tip:}
Best practice would be to use a file naming convention that includes a version and/or date.
\end{sphinxadmonition}
\end{quote}

\end{enumerate}


\subsubsection{Install Your Package}
\label{\detokenize{usage:install-your-package}}\begin{enumerate}
\item {} 
Connect to your terminal’s main serial port using \sphinxstylestrong{Console} from the \sphinxstylestrong{Developer Toolkit}.  Ensure your terminal is powered up.

\item {} 
Select \sphinxmenuselection{File \(\rightarrow\) Open \(\rightarrow\) Open Connection} then select your \sphinxstylestrong{COM} port and Baud rate (default 9600).  Click \sphinxmenuselection{OK}.

\item {} 
Confirm that you see a \sphinxcode{\sphinxupquote{shell\textgreater{}}}, \sphinxcode{\sphinxupquote{\#}}, or \sphinxcode{\sphinxupquote{boot\#}} prompt.
\begin{quote}

\begin{sphinxadmonition}{tip}{Tip:}
You may need to power cycle the terminal to get the \sphinxcode{\sphinxupquote{boot loader}} message, then send 2 \sphinxstyleemphasis{break} (\sphinxcode{\sphinxupquote{CTRL-B}}) commands within 5 seconds to get the \sphinxcode{\sphinxupquote{boot\#}} prompt.
\end{sphinxadmonition}
\end{quote}

\item {} 
Drag and drop your \sphinxstylestrong{build package} onto the black area of the screen.  Confirm that you see it begin the load process.
\begin{quote}
\begin{itemize}
\item {} 
If prompted to \sphinxstyleemphasis{Format the flash file system}, select \sphinxmenuselection{OK}.

\end{itemize}

\begin{figure}[htbp]
\centering
\capstart

\noindent\sphinxincludegraphics[scale=0.5]{{terminal-formatflash}.jpg}
\caption{Format the flash file system}\label{\detokenize{usage:id3}}\end{figure}
\begin{itemize}
\item {} 
Wait for the progress bar to complete.

\end{itemize}

\begin{figure}[htbp]
\centering
\capstart

\noindent\sphinxincludegraphics[scale=0.5]{{terminal-packageload}.jpg}
\caption{Loading the package file using Console}\label{\detokenize{usage:id4}}\end{figure}
\begin{itemize}
\item {} 
If you see the \sphinxcode{\sphinxupquote{\#}} prompt, enter the command \sphinxcode{\sphinxupquote{start}}.

\end{itemize}
\end{quote}

\item {} 
If you are running Modbus on the main rs232, you will see a \sphinxcode{\sphinxupquote{detaching shell}} message, then strange characters.
\begin{quote}

\begin{sphinxadmonition}{tip}{Tip:}
You can set up trace/log output on a spare serial port of the terminal.  For example, use rs232 for trace when rs485 is connected to your Modbus device.
\end{sphinxadmonition}
\end{quote}

\item {} 
Connect the configured serial port to your Modbus device.

\end{enumerate}


\subsubsection{Create Message Definition File}
\label{\detokenize{usage:create-message-definition-file}}\begin{enumerate}
\item {} 
Contact your \sphinxstyleemphasis{Inmarsat Representative} to send your \sphinxstylestrong{idpmsg} file to have merged with default message definitions.

\end{enumerate}


\subsubsection{Uplode Message Definition File to Inmarsat Message Gateway}
\label{\detokenize{usage:uplode-message-definition-file-to-inmarsat-message-gateway}}\begin{enumerate}
\item {} 
Your service provider can upload/associate the \sphinxstylestrong{message definition file} to your IsatData Pro \sphinxstylestrong{mailbox} and provide your authorization credentials for \sphinxstylestrong{OTA API} use.

\end{enumerate}


\subsubsection{Retrieve Data Using the IsatData Pro Messaging API}
\label{\detokenize{usage:retrieve-data-using-the-isatdata-pro-messaging-api}}\begin{enumerate}
\item {} 
Visit \sphinxhref{https://developer.inmarsat.com}{Inmarsat Developer Portal}

\end{enumerate}


\section{Configuration}
\label{\detokenize{configuration:configuration}}\label{\detokenize{configuration::doc}}
The Modbus service implements 3 primary functional modules: communication, reporting and analytics.  Configuration of all three modules is done using any of:
\begin{itemize}
\item {} 
\sphinxcode{\sphinxupquote{config.dat}} file loaded locally on the console

\item {} 
console shell commands

\item {} 
\sphinxstylestrong{OTA API}

\item {} 
\sphinxstylestrong{LSF API}

\end{itemize}

Additionally, optional functional modules support geolocation movement alerts, power-saving operation, and diagnostics logs.


\subsection{The Configuration File}
\label{\detokenize{configuration:the-configuration-file}}\label{\detokenize{configuration:configfile}}
The most practical way of configuring for deployment is to create a \sphinxcode{\sphinxupquote{config.dat}} file and build it into a \sphinxstylestrong{.idppkg} package file.

\sphinxstylestrong{Comments} can be added to the \sphinxcode{\sphinxupquote{config.dat}} file by enclosing in \sphinxcode{\sphinxupquote{/*}} and \sphinxcode{\sphinxupquote{*/}}.

A simple example \sphinxcode{\sphinxupquote{config.dat}} file could look like {\hyperref[\detokenize{usage:example-config-dat}]{\sphinxcrossref{\DUrole{std,std-ref}{this}}}}.


\subsection{Shell Commands}
\label{\detokenize{configuration:shell-commands}}
For initial testing of the service, you can use \sphinxstyleemphasis{shell} commands on the main serial port of the terminal to configure individual \sphinxstylestrong{ports}, \sphinxstylestrong{devices}, \sphinxstylestrong{parameters}, \sphinxstylestrong{reports} and \sphinxstylestrong{alerts}.

The \sphinxstyleemphasis{shell} commands can be queried iteratively using the \sphinxcode{\sphinxupquote{?}} operator:

\fvset{hllines={, ,}}%
\begin{sphinxVerbatim}[commandchars=\\\{\}]
modbus ?
\end{sphinxVerbatim}

An example \sphinxstyleemphasis{shell} command to configure a \sphinxstylestrong{parameter} would look like:

\fvset{hllines={, ,}}%
\begin{sphinxVerbatim}[commandchars=\\\{\}]
\PYG{n}{modbus} \PYG{n}{config} \PYG{n+nb}{set} \PYG{n}{parameter} \PYG{n}{paramId}\PYG{o}{=}\PYG{l+m+mi}{1} \PYG{n}{deviceId}\PYG{o}{=}\PYG{l+m+mi}{1} \PYG{n}{registerType}\PYG{o}{=}\PYG{n}{analog} \PYG{n}{address}\PYG{o}{=}\PYG{l+m+mi}{31} \PYG{n}{encoding}\PYG{o}{=}\PYG{n}{int16}
\end{sphinxVerbatim}


\subsection{Communications Configuration}
\label{\detokenize{configuration:communications-configuration}}
The Modbus service is typically configured to interface with a single industrial device, but can support connection to a Modbus network on each available serial port of the satellite terminal.  Modbus is a master-slave protocol, and the satellite terminal acts as the master and polls device slave(s) attached to the RS232 and/or RS485 serial port.

The communication hierarchy to be configured is:
\begin{itemize}
\item {} 
\sphinxcode{\sphinxupquote{port}} defines the physical serial port used to connect to the Modbus device or network.

\item {} 
\sphinxcode{\sphinxupquote{device}} is a Modbus slave on a given Port, assigned a unique ID within the service context.  Multiple Devices can be associated with a single Port.

\item {} 
\sphinxcode{\sphinxupquote{parameter}} associates to an individual data value obtained from Modbus register(s) on a given Device.

\item {} 
\sphinxcode{\sphinxupquote{property}} \sphinxstyleemphasis{optionally} associates to a \sphinxstylestrong{parameter} to expose that Modbus data value individually to the \sphinxstylestrong{OTA API} and/or \sphinxstylestrong{LSF API}.

\end{itemize}


\subsubsection{Ports}
\label{\detokenize{configuration:ports}}\label{\detokenize{configuration:id1}}
The satellite terminal supports one or more serial ports depending on the hardware model (see Terminal Reference Table). Each \sphinxcode{\sphinxupquote{port}} can be configured to talk to a Modbus slave device or, in the case of RS485 multi-drop, a network of devices.

Each \sphinxstylestrong{port} has the following parameters:
\begin{itemize}
\item {} 
\sphinxcode{\sphinxupquote{port}} (\sphinxstyleemphasis{required})
\begin{quote}

\sphinxcode{\sphinxupquote{rs232}} or \sphinxcode{\sphinxupquote{rs232aux}} or \sphinxcode{\sphinxupquote{rs485}} depending on the physical port available on the terminal:


\begin{savenotes}\sphinxattablestart
\centering
\sphinxcapstartof{table}
\sphinxcaption{Ports available on satellite terminal models}\label{\detokenize{configuration:id5}}
\sphinxaftercaption
\begin{tabular}[t]{|*{4}{\X{1}{4}|}}
\hline

\sphinxstylestrong{Terminal Model}
&
\sphinxstylestrong{rs232}
&
\sphinxstylestrong{rs232aux}
&
\sphinxstylestrong{rs485}
\\
\hline
ST6100
&\begin{itemize}
\item {} 
\end{itemize}
&&\begin{itemize}
\item {} 
\end{itemize}
\\
\hline
IDP-680/690
&\begin{itemize}
\item {} 
\end{itemize}
&&\begin{itemize}
\item {} 
\end{itemize}
\\
\hline
IDP-782
&\begin{itemize}
\item {} 
\end{itemize}
&\begin{itemize}
\item {} 
\end{itemize}
&\begin{itemize}
\item {} 
\end{itemize}
\\
\hline
IDP-800
&\begin{itemize}
\item {} 
\end{itemize}
&&\\
\hline
\end{tabular}
\par
\sphinxattableend\end{savenotes}
\end{quote}

\item {} 
\sphinxcode{\sphinxupquote{baudRate}} (\sphinxstyleemphasis{required})
\begin{quote}

A number value in the range \sphinxcode{\sphinxupquote{1200}}, \sphinxcode{\sphinxupquote{2400}}, \sphinxcode{\sphinxupquote{4800}}, \sphinxcode{\sphinxupquote{9600}}, \sphinxcode{\sphinxupquote{19200}}, \sphinxcode{\sphinxupquote{38400}}, \sphinxcode{\sphinxupquote{57600}}, \sphinxcode{\sphinxupquote{115200}}
\end{quote}

\item {} 
\sphinxcode{\sphinxupquote{parity}} (\sphinxstyleemphasis{required})
\begin{quote}

\sphinxcode{\sphinxupquote{even}}, \sphinxcode{\sphinxupquote{odd}} or \sphinxcode{\sphinxupquote{none}}
\end{quote}

\item {} 
\sphinxcode{\sphinxupquote{mode}} (\sphinxstyleemphasis{required})
\begin{quote}

The variant of Modbus protocol used: \sphinxcode{\sphinxupquote{rtu}} or \sphinxcode{\sphinxupquote{ascii}}

\begin{sphinxadmonition}{note}{Note:}
The selection of \sphinxcode{\sphinxupquote{ascii}} or \sphinxcode{\sphinxupquote{rtu}} mode does not affect data transmission costs.
\end{sphinxadmonition}
\end{quote}

\end{itemize}


\subsubsection{Devices}
\label{\detokenize{configuration:devices}}\label{\detokenize{configuration:id2}}
Each Modbus device attached to a serial port is assigned a unique ID within the service.  If interfacing to multiple Modbus devices on a multi-drop RS485 network, each \sphinxcode{\sphinxupquote{device}} will have a “network” ID unique on that multi-drop network.

Each \sphinxstylestrong{device} has the following configuration parameters:
\begin{itemize}
\item {} 
\sphinxcode{\sphinxupquote{deviceId}} (\sphinxstyleemphasis{required})
\begin{quote}

A number from \sphinxstylestrong{1..255}, globally unique identifier for the device within the service, used to identify a physical device on the attached Modbus network.
Note this may be different from the Modbus \sphinxstyleemphasis{Device ID} (\sphinxstyleemphasis{slave address}) configured on the physical device.
\end{quote}

\item {} 
\sphinxcode{\sphinxupquote{networkId}} (\sphinxstyleemphasis{required})
\begin{quote}

A number from \sphinxstylestrong{1..247} mapping to the Modbus slave address of the physical device on the Modbus network.
Typically, this will be specified by the device manufacturer and often uses a default value of 1.
\end{quote}

\item {} 
\sphinxcode{\sphinxupquote{port}} (\sphinxstyleemphasis{required})
\begin{quote}

Serial communication \sphinxstylestrong{port} on which the Modbus Slave device is connected \sphinxcode{\sphinxupquote{rs232}}, \sphinxcode{\sphinxupquote{rs232aux}} or \sphinxcode{\sphinxupquote{rs485}}.
\end{quote}

\item {} 
\sphinxcode{\sphinxupquote{pollInterval}} (\sphinxstyleemphasis{required})
\begin{quote}

Interval in seconds from \sphinxstylestrong{0..604800} to poll the configured set of register data from a particular device.
\begin{itemize}
\item {} 
0 = disabled

\item {} 
604800 = weekly

\end{itemize}

\begin{sphinxadmonition}{note}{Note:}
This value by itself does not affect satellite data use, and typically this is configured at a fast interval in order to capture value changes in real-time to supplement periodic \sphinxstylestrong{reports} with \sphinxstylestrong{alerts}.
\end{sphinxadmonition}
\end{quote}

\item {} 
\sphinxcode{\sphinxupquote{byteOrder}} (\sphinxstyleemphasis{required})
\begin{quote}

\sphinxcode{\sphinxupquote{msb}} (\sphinxstyleemphasis{most significant bit}) or \sphinxcode{\sphinxupquote{lsb}} (\sphinxstyleemphasis{least significant bit}) is a sequential order in which bytes are sent over serial link, and is generally defined by the Modbus device manufacturer (based on the microprocessor they use).

\begin{sphinxadmonition}{note}{Note:}
Most Significant Bit (msb) format is also called big-endian or network byte order, and is the typical/default method used.
\end{sphinxadmonition}
\end{quote}

\item {} 
\sphinxcode{\sphinxupquote{wordOrder}} (\sphinxstyleemphasis{required})
\begin{quote}

\sphinxcode{\sphinxupquote{msw}} (\sphinxstyleemphasis{most significant word}) or \sphinxcode{\sphinxupquote{lsw}} (\sphinxstyleemphasis{least signficant word}) similar to the \sphinxstyleemphasis{byteOrder} principle, Modbus registers are stored as 16-bit \sphinxstyleemphasis{words} and the wordOrder is a sequential order in which words, formed by bytes, are to be interpreted to read or write the data value.
When polling a block of contiguous registers, the word order is important to be able to interpret the data, for example to convert to a standard integer number for transmission or analytics.
\end{quote}

\item {} 
\sphinxcode{\sphinxupquote{serialRxTimeout}} (\sphinxstyleemphasis{optional})
\begin{quote}

Time in milliseconds from \sphinxstylestrong{100..30000} to wait for read data from the Modbus device.
If the Modbus device fails to send a byte serialRxTimeout milliseconds after the last byte, a read timeout error will be raised.
\end{quote}

\item {} 
\sphinxcode{\sphinxupquote{serialTxTimeout}} (\sphinxstyleemphasis{optional})
\begin{quote}

Time in milliseconds from \sphinxstylestrong{100..30000} to wait for the acknowledgement of transmission to the Modbus device.
If the Modbus device fails to send an acknowledgement within serialTx Timeout milliseconds after the last byte has been sent, a write timeout error will be raised.
\end{quote}

\item {} 
\sphinxcode{\sphinxupquote{plcBaseAddress}} (\sphinxstyleemphasis{optional})
\begin{quote}

A flag value set to \sphinxstylestrong{1} if the Modbus device does not use zero-based addressing.
Some Modbus devices (typically Programmable Logic Controllers \sphinxstylestrong{PLC}) define register addressing with an offset of 1 (e.g. register 1 on the PLC is actually data address 0).

Setting plcBaseAddress equal to 1 allows the configuration of Parameters that correspond directly to the device manufacturer’s documentation.
\end{quote}

\item {} 
\sphinxcode{\sphinxupquote{retries}} (\sphinxstyleemphasis{optional})
\begin{quote}

The number of retries \sphinxstylestrong{1..10} that determines how many data transmission attempts will be made before raising an error. The default value is 1.
\end{quote}

\end{itemize}


\subsubsection{Parameters}
\label{\detokenize{configuration:parameters}}\label{\detokenize{configuration:id3}}
Each \sphinxstylestrong{parameter} is a uniquely identified data value that maps to a Modbus register value or contiguous register range of a common type used to define a single value.
Certain data types support analytics that can provide filtering and escalation of critical changes to supplement conventional polled data collected by the centralized application.

A \sphinxstylestrong{parameter} can also be mapped to a \sphinxstylestrong{property} {\hyperref[\detokenize{configuration:properties}]{\sphinxcrossref{\DUrole{std,std-ref}{Properties}}}}, which are exported for direct access by either the \sphinxstylestrong{OTA API} or \sphinxstylestrong{LSF API}.

Parameters have the following configurable attributes:
\begin{itemize}
\item {} 
\sphinxcode{\sphinxupquote{paramId}} (\sphinxstyleemphasis{required})
\begin{quote}

A number from \sphinxstylestrong{1..255} used to uniquely identify this particular device and parameter within the service for the data value of interest.
\end{quote}

\item {} 
\sphinxcode{\sphinxupquote{deviceId}} (\sphinxstyleemphasis{required})
\begin{quote}

A number from \sphinxstylestrong{1..255} mapping to the {\hyperref[\detokenize{configuration:devices}]{\sphinxcrossref{\DUrole{std,std-ref}{Device}}}} the parameter is associated with.
\end{quote}

\item {} 
\sphinxcode{\sphinxupquote{registerType}} (\sphinxstyleemphasis{required})
\begin{quote}

A register type \sphinxcode{\sphinxupquote{coil}}, \sphinxcode{\sphinxupquote{input}}, \sphinxcode{\sphinxupquote{analog}} or \sphinxcode{\sphinxupquote{holding}}.

Typically Modbus registers are defined based on the following ranges:


\begin{savenotes}\sphinxattablestart
\centering
\sphinxcapstartof{table}
\sphinxcaption{Modbus Register Ranges and Types}\label{\detokenize{configuration:id6}}
\sphinxaftercaption
\begin{tabulary}{\linewidth}[t]{|T|T|T|T|}
\hline

Register Number
&
Data Address
&
Modbus Definition
&
\sphinxcode{\sphinxupquote{registerType}}
\\
\hline
1 .. 9999
&
0x0000 .. 0x270E
&
Discrete Output Coils
&
\sphinxcode{\sphinxupquote{coil}}
\\
\hline
10001 .. 19999
&
0x0000 .. 0x270E
&
Discrete Input Contacts
&
\sphinxcode{\sphinxupquote{input}}
\\
\hline
30001 .. 39999
&
0x0000 .. 0x270E
&
Analog Input Registers
&
\sphinxcode{\sphinxupquote{analog}}
\\
\hline
40001 .. 49999
&
0x0000 .. 0x270E
&
Holding Registers (Analog Output)
&
\sphinxcode{\sphinxupquote{holding}}
\\
\hline
\end{tabulary}
\par
\sphinxattableend\end{savenotes}
\end{quote}

\item {} 
\sphinxcode{\sphinxupquote{encoding}} (\sphinxstyleemphasis{required})
\begin{quote}

Defines how to interpret data from Modbus registers for analytics or reporting.
\begin{itemize}
\item {} 
\sphinxcode{\sphinxupquote{boolean}} converts 0 value to false and any non-zero value to true

\item {} 
\sphinxcode{\sphinxupquote{int8}} converts register data to 8-bit signed integer value from -128..127

\item {} 
\sphinxcode{\sphinxupquote{uint8}} converts register data to 8-bit unsigned integer value from 0..255

\item {} 
\sphinxcode{\sphinxupquote{int16}} converts register data to 16-bit signed integer value from -32768..32767

\item {} 
\sphinxcode{\sphinxupquote{uint16}} converts register data to 8-bit unsigned integer value from 0..65535

\item {} 
\sphinxcode{\sphinxupquote{int32}} converts register data to 32-bit signed integer value from -2147483648..2147483647

\end{itemize}

\begin{sphinxadmonition}{note}{Note:}
32-bit unsigned integer is not supported by some versions of the \sphinxstylestrong{LSF} and therefore is not supported by the Modbus service.
\end{sphinxadmonition}
\begin{itemize}
\item {} 
\sphinxcode{\sphinxupquote{float32}} converts register data to a single precision floating number.

\end{itemize}

\begin{sphinxadmonition}{note}{Note:}
Floating point numbers are sent as \sphinxstylestrong{string} data types in the \sphinxstylestrong{OTA API}
\end{sphinxadmonition}
\begin{itemize}
\item {} 
\sphinxcode{\sphinxupquote{string}} converts register data into an ASCII format

\item {} 
\sphinxcode{\sphinxupquote{base64string}} converts register data into a base64 string format

\item {} 
\sphinxcode{\sphinxupquote{raw}} converts register data into a single byte array format

\end{itemize}
\end{quote}

\item {} 
\sphinxcode{\sphinxupquote{address}} (\sphinxstyleemphasis{required})
\begin{quote}

The \sphinxstyleemphasis{data address} or from \sphinxstylestrong{0..65535} for the starting register to read.

\begin{sphinxadmonition}{note}{Note:}
If values are not reading properly from your Modbus device, you may need to set \sphinxcode{\sphinxupquote{plcBaseAddress}} to 1.
\end{sphinxadmonition}
\end{quote}

\item {} 
\sphinxcode{\sphinxupquote{length}} (\sphinxstyleemphasis{conditional})
\begin{quote}

The number of registers to read from \sphinxstylestrong{1..125} holding a single data value.
Since Modbus registers are 16-bits, length is implicitly=1 for encodingType up to 16 bits, and implicitly=2 for 32-bit encodingType.

Length is \sphinxstyleemphasis{required} to be specified when encoding is configured as \sphinxcode{\sphinxupquote{string}}, \sphinxcode{\sphinxupquote{base64string}} or \sphinxcode{\sphinxupquote{raw}}.
\end{quote}

\item {} 
\sphinxcode{\sphinxupquote{mult}} (\sphinxstyleemphasis{optional})
\begin{quote}

A multiplier from \sphinxstylestrong{1..100} applied when reading the encoded register value, typically for purposes of analytics, reporting or API access.

\begin{sphinxadmonition}{note}{Note:}
Conversion of floating point to a whole number is useful for analytics since floating point math is problematic in the \sphinxstylestrong{LSF}.
\end{sphinxadmonition}
\end{quote}

\end{itemize}


\subsubsection{Properties}
\label{\detokenize{configuration:properties}}\label{\detokenize{configuration:id4}}
Properties are \sphinxstyleemphasis{optional} and intended to be used within the \sphinxstylestrong{LSF API} to directly read or write configuration settings and data parameters from the local console, over-the-air messaging interface or other LSF services, accessed by service-specific Property Identification Numbers (PIN).  The Modbus service allows you to assign an individual device data parameter to a PIN that is exported for access by the LSF.

Properties have the following configurable attributes:
\begin{itemize}
\item {} 
\sphinxcode{\sphinxupquote{paramId}} (\sphinxstyleemphasis{required})
\begin{quote}

The unique {\hyperref[\detokenize{configuration:parameters}]{\sphinxcrossref{\DUrole{std,std-ref}{Parameters}}}} id from \sphinxstylestrong{1..255} corresponding to the specific device and register value of interest.
\end{quote}

\item {} 
\sphinxcode{\sphinxupquote{pin}} (\sphinxstyleemphasis{required})
\begin{quote}

The unique Property Identifier Number from \sphinxstylestrong{10..255} where the parameter value can be read as a \sphinxstylestrong{property}.
\end{quote}

\end{itemize}


\subsection{Reporting Configuration}
\label{\detokenize{configuration:reporting-configuration}}
A \sphinxstylestrong{report} is used to send one or more \sphinxstylestrong{parameter} values on a periodic basis.
The report interval can be stochastic based on an offset from when the terminal is powered up or reset, or based on time of day.

\begin{sphinxadmonition}{note}{Note:}
Time of day reports \sphinxstyleemphasis{must} be distributed over a population of terminals to minimize satellite packet loss and retransmissions that could affect network quality of service.
To accommodate fixed time reports, the defined Modbus registers are read at the configured time of day as a \sphinxstyleemphasis{snapshot}, but the report is distributed over a configurable window of time.
A given terminal will always report at the same time offset within the window, but any 2 terminals will have different offsets within the window.
\end{sphinxadmonition}


\subsubsection{Reports}
\label{\detokenize{configuration:reports}}
Reports are configured either locally via a \sphinxcode{\sphinxupquote{config.dat}} file, manual console typed commands, \sphinxstylestrong{OTA API} or \sphinxstylestrong{LSF API}.

Reports have the following configurable attributes:
\begin{itemize}
\item {} 
\sphinxcode{\sphinxupquote{reportingId}} (\sphinxstyleemphasis{required})
\begin{quote}

A unique identifier from \sphinxstylestrong{1..255} corresponding to a defined set of \sphinxstylestrong{parameters}.
\end{quote}

\item {} \begin{description}
\item[{\sphinxcode{\sphinxupquote{paramIds}} (\sphinxstyleemphasis{required})}] \leavevmode
A string of comma separated values of unique \sphinxstylestrong{parameter id(s)} that will be included in the report.

\end{description}

\item {} 
\sphinxcode{\sphinxupquote{interval}} (\sphinxstyleemphasis{required})
\begin{quote}

The frequency of transmission in seconds from \sphinxstylestrong{0..604800}.  The timer starts as soon as the configuration is processed, or each terminal reboot.

\begin{DUlineblock}{0em}
\item[] 0 = disabled (do not send)
\item[] 604800 = weekly
\end{DUlineblock}
\end{quote}

\item {} 
\sphinxcode{\sphinxupquote{timeOfDay}} (\sphinxstyleemphasis{optional})
\begin{quote}

The time of day in seconds from midnight UTC \sphinxstylestrong{0..86400} when a \sphinxstyleemphasis{snapshot} reading of the associated \sphinxstylestrong{parameters} will be done.

\begin{DUlineblock}{0em}
\item[] 0 = disabled (do not send)
\item[] 86400 \textendash{} midnight (starting next day)
\end{DUlineblock}

\begin{sphinxadmonition}{note}{Note:}
The time of day report will be sent \sphinxstyleemphasis{after} the measurement, witin a window specified by \sphinxcode{\sphinxupquote{timeOfDayWindow}}.
\end{sphinxadmonition}
\end{quote}

\item {} 
\sphinxcode{\sphinxupquote{timeOfDayWindow}} (\sphinxstyleemphasis{conditional})
\begin{quote}

A time period in seconds from \sphinxstylestrong{0..3600} that must be configured if using \sphinxcode{\sphinxupquote{timeOfDay}} is non-zero.
This is used to delay reports that are scheduled to \sphinxstyleemphasis{snapshot} at a specific time of day, to avoid satellite network congestion.

The \sphinxstyleemphasis{window} is established immediately after the scheduled timeOfDay during which a given terminal can report its snapshot parameters.

\begin{sphinxadmonition}{note}{Note:}
A given terminal will transmit at the same offset each period.  But two different terminals with same \sphinxcode{\sphinxupquote{timeOfDay}} may transmit at different times.
\end{sphinxadmonition}
\end{quote}

\item {} 
\sphinxcode{\sphinxupquote{msgBitmap}} (\sphinxstyleemphasis{optional})
\begin{quote}

An ASCII hex string value from \sphinxstylestrong{“00”..”FF”} used to override the default Bitmap and include/exclude optional message fields in this particular \sphinxstylestrong{report}.

See: {\hyperref[\detokenize{configuration:msgbitmap}]{\sphinxcrossref{\DUrole{std,std-ref}{msgBitmap}}}}
\end{quote}

\end{itemize}


\subsection{Analytics Configuration}
\label{\detokenize{configuration:analytics-configuration}}
Analytics configurations allow creating \sphinxstylestrong{alerts}, configured individually for a critical parameter based on the following trigger types:
\begin{itemize}
\item {} 
High threshold value exceeded

\item {} 
Low threshold value exceeded

\item {} 
Return-to-midrange from High or Low threshold

\item {} 
An relative change of value

\end{itemize}


\subsubsection{Alerts}
\label{\detokenize{configuration:alerts}}
\sphinxstylestrong{Alerts} are configured via a \sphinxcode{\sphinxupquote{config.dat}} file, manually via console commands, \sphinxstylestrong{OTA API} or \sphinxstylestrong{LSF API}, based on a simple criteria against a specific \sphinxstylestrong{parameter} defined by a \sphinxcode{\sphinxupquote{paramId}}.

\begin{sphinxadmonition}{note}{Note:}
Only one \sphinxstylestrong{alert} can be defined against a given \sphinxstylestrong{paramId}, to avoid ambiguity of conditions.
\end{sphinxadmonition}

In order to capture analog measurement values that may fluctuate within a small range, the minimum and maximum threshold values can be configured with bounds using on/off qualifiers.
For example, consider a wind measurement in which we want an \sphinxstylestrong{alert} when the wind goes above a safe range of 45 km/h, but persist the alert condition until the wind drops below 38 km/h.
Assuming the Modbus register for wind speed is presented in tenths of km/h, we set maxON=450 and maxOFF=380.

Analytics have the following configurable attributes:
\begin{itemize}
\item {} 
\sphinxcode{\sphinxupquote{alertId}} (\sphinxstyleemphasis{required})
\begin{quote}

A unique identifier from \sphinxstylestrong{1..255} for a particular analytics configuration indicated in the \sphinxstylestrong{alert} message.
\end{quote}

\item {} 
\sphinxcode{\sphinxupquote{paramId}} (\sphinxstyleemphasis{required})
\begin{quote}

The unique \sphinxstylestrong{parameter} ID from \sphinxstylestrong{1..255} used as the \sphinxstylestrong{alert} trigger.
This parameter’s value will be included in the \sphinxstylestrong{alert} message by default.
\end{quote}

\item {} 
\sphinxcode{\sphinxupquote{paramIds}} (\sphinxstyleemphasis{optional})
\begin{quote}

A string of comma separated values of unique \sphinxstylestrong{parameter id(s)} that will be included in the alert, in addition to the trigger value.
This is typically used to cross-reference relevant data sources affected by the trigger value or as useful metadata to the alert condition.
\end{quote}

\item {} 
\sphinxcode{\sphinxupquote{minON}} (\sphinxstyleemphasis{optional})
\begin{quote}

A lower threshold value from \sphinxstylestrong{-2147483647..2147483647} against which the parameter value will be tested.
If the parameter value drops \sphinxstyleemphasis{below minON}, a minimum ON alert {\hyperref[\detokenize{otaapi:paramalerton}]{\sphinxcrossref{\DUrole{std,std-ref}{paramAlertON (MIN 101)}}}} will be generated, indicating that current value crossed the minimum threshold.

This alarm condition will persist until the value rises \sphinxstyleemphasis{above minOFF}.
\end{quote}

\item {} 
\sphinxcode{\sphinxupquote{minOFF}} (\sphinxstyleemphasis{optional})
\begin{quote}

A mid-range threshold value from \sphinxstylestrong{-2147483647..2147483647} against which the parameter value will be tested.
If the parameter value raises \sphinxstyleemphasis{above minOFF} after having dropped below minON, a minimum OFF alert {\hyperref[\detokenize{otaapi:paramalertoff}]{\sphinxcrossref{\DUrole{std,std-ref}{paramAlertOFF (MIN 102)}}}} will be generated, indicating that current value returned to the normal range.
\end{quote}

\item {} 
\sphinxcode{\sphinxupquote{maxON}} (\sphinxstyleemphasis{optional})
\begin{quote}

An upper threshold value from \sphinxstylestrong{-2147483647..2147483647} against which the parameter value will be tested.
If the parameter value raises \sphinxstyleemphasis{above maxON}, a maximum ON alert {\hyperref[\detokenize{otaapi:paramalerton}]{\sphinxcrossref{\DUrole{std,std-ref}{paramAlertON (MIN 101)}}}} will be generated, indicating that current value crossed the maximum threshold.

This alarm condition will persist until the value drops \sphinxstyleemphasis{below maxOFF}.
\end{quote}

\item {} 
\sphinxcode{\sphinxupquote{maxOFF}} (\sphinxstyleemphasis{optional})
\begin{quote}

A mid-range threshold value from \sphinxstylestrong{-2147483647..2147483647} against which the parameter value will be tested.
If the parameter value drops \sphinxstyleemphasis{below maxOFF} after having raised above minON, a maximum OFF alert {\hyperref[\detokenize{otaapi:paramalertoff}]{\sphinxcrossref{\DUrole{std,std-ref}{paramAlertOFF (MIN 102)}}}} will be generated, indicating that current value returned to the normal range.
\end{quote}

\item {} 
\sphinxcode{\sphinxupquote{change}} (\sphinxstyleemphasis{optional})
\begin{quote}

A relative threshold value from \sphinxstylestrong{0..2147483647} against which the last reported parameter value will be tested.
If the parameter value changes by this value or more relative to the last reported value, an alert is generated.
\end{quote}

\item {} 
\sphinxcode{\sphinxupquote{msgBitmap}} (\sphinxstyleemphasis{optional})
\begin{quote}

An ASCII hex string value from \sphinxstylestrong{“00”..”FF”} used to override the default Bitmap and include/exclude optional message fields in this particular \sphinxstylestrong{alert}.

See: {\hyperref[\detokenize{configuration:msgbitmap}]{\sphinxcrossref{\DUrole{std,std-ref}{msgBitmap}}}}
\end{quote}

\end{itemize}


\subsection{Default Settings (Configuration Properties)}
\label{\detokenize{configuration:default-settings-configuration-properties}}
A \sphinxstylestrong{LSF} service contains configuration \sphinxstylestrong{properties} each referenced by unique Property Identification Number (\sphinxstylestrong{pin}) within the scope of the service (\sphinxstylestrong{sin}).

\begin{sphinxadmonition}{note}{Note:}
This section assumes the reader has a good understanding of the ORBCOMM Lua Service Framework and is familiar with how to read their Lua/Terminal API specification.
\end{sphinxadmonition}


\subsubsection{Geolocation Movement Alert Configuration}
\label{\detokenize{configuration:geolocation-movement-alert-configuration}}\label{\detokenize{configuration:geolocation-config}}
By default, \sphinxstyleemphasis{geolocation} alerts are disabled.  To enable \sphinxstyleemphasis{geolocation} alerts, you must configure user service property \sphinxcode{\sphinxupquote{assetMoveDist}} with a non-zero distance (in meters).
You can also change the user service property \sphinxcode{\sphinxupquote{posAlarmDebounce}} from its default value to modify the \sphinxstyleemphasis{geolocation} behaviour.


\subsubsection{LSF Properties}
\label{\detokenize{configuration:lsf-properties}}

\begin{savenotes}\sphinxattablestart
\centering
\sphinxcapstartof{table}
\sphinxcaption{Configuration Properties}\label{\detokenize{configuration:id7}}\label{\detokenize{configuration:config-properties}}
\sphinxaftercaption
\begin{tabular}[t]{|*{6}{\X{1}{6}|}}
\hline
\sphinxstyletheadfamily 
PIN
&\sphinxstyletheadfamily 
Name
&\sphinxstyletheadfamily 
Description
&\sphinxstyletheadfamily 
Type
&\sphinxstyletheadfamily 
Storage Class
&\sphinxstyletheadfamily 
Default Value
\\
\hline
1
&
logsBitmap
&
Hexadecimal binary mask to enable/disable logs.
\begin{itemize}
\item {} 
Errors

\item {} 
Warnings

\item {} 
Debug

\end{itemize}
&
string
&
Config
&
E0 (error/warning)
\\
\hline
3
&
lostComMsgDebounce
&
Duration (seconds) to ignore subsequent communication losses after sending an \sphinxstyleemphasis{alert}.
&
unsignedInt
&
Config
&
120
\\
\hline
4
&
defaultMsgBitmap
&
Hexadecimal binary mask to include metadata fields in \sphinxstylestrong{reports} and \sphinxstylestrong{alerts}.
\begin{itemize}
\item {} 
latitude

\item {} 
longitude

\item {} 
speed

\item {} 
altitude

\item {} 
distance

\item {} 
timestamp

\item {} 
paramTimestamp

\item {} 
(reserved)

\end{itemize}
&
string
&
Config
&
FF (all enabled)
\\
\hline
5
&
changeAlarmDebounce
&
Duration (seconds) to ignore subsequent \sphinxstyleemphasis{change} detections after sending an \sphinxstylestrong{alert}.
&
\begin{DUlineblock}{0em}
\item[] unsignedInt
\item[] max=3600
\end{DUlineblock}
&
Config
&
120
\\
\hline
6
&
posAlarmDebounce
&
Duration (seconds) to ignore subsequent \sphinxstyleemphasis{geolocation} detections after sending an \sphinxstylestrong{alert}.
&
\begin{DUlineblock}{0em}
\item[] unsignedInt
\item[] max=86400
\end{DUlineblock}
&
Config
&
3600
\\
\hline
7
&
isLPM
&
Enables power saving operation.
&
boolean
&
Config
&
false
\\
\hline
8
&
refLatitude
&
Latitude (milliminutes) used as a reference point for \sphinxstyleemphasis{geolocation} alerts.
&
\begin{DUlineblock}{0em}
\item[] signedInt
\item[] min=-5400000
\item[] max=5400000
\end{DUlineblock}
&
Config
&
0
\\
\hline
9
&
refLongitude
&
Longitude (milliminutes) used as a reference point for \sphinxstyleemphasis{geolocation} alerts.
&
\begin{DUlineblock}{0em}
\item[] signedInt
\item[] min=-10800000
\item[] max=10800000
\end{DUlineblock}
&
Config
&
0
\\
\hline
10
&
assetMoveDist
&
Distance threshold (meters) used to trigger \sphinxstyleemphasis{geolocation} alerts.
&
\begin{DUlineblock}{0em}
\item[] unsignedInt
\item[] 0 = disabled
\end{DUlineblock}
&
Config
&
0
\\
\hline
11..255
&
Output1..Output245
&
Parameters values mapped to Properties
&
string
&
Volatile
&
“”
\\
\hline
\end{tabular}
\par
\sphinxattableend\end{savenotes}


\subsubsection{Optional Metadata Fields in Messages}
\label{\detokenize{configuration:optional-metadata-fields-in-messages}}
Verbose metadata is, by default, enabled on all \sphinxstylestrong{reports} and \sphinxstylestrong{alerts}.
You can change this behaviour by any of the following methods:
\begin{itemize}
\item {} 
Change the default for \sphinxstyleemphasis{ALL} \sphinxstylestrong{reports} and \sphinxstylestrong{alerts} using \sphinxcode{\sphinxupquote{defaultMsgBitmap}} property value in your {\hyperref[\detokenize{usage:solution-studio-package}]{\sphinxcrossref{\DUrole{std,std-ref}{build package}}}}

\item {} 
Configure on a per-report or per-alert basis using your {\hyperref[\detokenize{configuration:configfile}]{\sphinxcrossref{\DUrole{std,std-ref}{configuration file}}}}

\item {} 
or using the {\hyperref[\detokenize{otaapi:to-mobile-messages}]{\sphinxcrossref{\DUrole{std,std-ref}{OTA API}}}}

\end{itemize}


\begin{savenotes}\sphinxattablestart
\centering
\sphinxcapstartof{table}
\sphinxcaption{Optional Metadata Fields Configuration (MsgBitmap, defaultMsgBitmap)}\label{\detokenize{configuration:id8}}\label{\detokenize{configuration:msgbitmap}}
\sphinxaftercaption
\begin{tabular}[t]{|\X{15}{36}|\X{7}{36}|\X{7}{36}|\X{7}{36}|}
\hline
\sphinxstyletheadfamily 
Optional Field
&\sphinxstyletheadfamily 
Default Setting
&\sphinxstyletheadfamily 
Example 1 (lat/lng/alt/time)
&\sphinxstyletheadfamily 
Example 2 (time only)
\\
\hline
latitude
&\begin{itemize}
\item {} 
\end{itemize}
&\begin{itemize}
\item {} 
\end{itemize}
&\\
\hline
longitude
&\begin{itemize}
\item {} 
\end{itemize}
&\begin{itemize}
\item {} 
\end{itemize}
&\\
\hline
speed
&\begin{itemize}
\item {} 
\end{itemize}
&&\\
\hline
altitude
&\begin{itemize}
\item {} 
\end{itemize}
&\begin{itemize}
\item {} 
\end{itemize}
&\\
\hline
distance
&\begin{itemize}
\item {} 
\end{itemize}
&&\\
\hline
timestamp
&\begin{itemize}
\item {} 
\end{itemize}
&\begin{itemize}
\item {} 
\end{itemize}
&\begin{itemize}
\item {} 
\end{itemize}
\\
\hline
paramTimestamp
&\begin{itemize}
\item {} 
\end{itemize}
&&\\
\hline
(\sphinxstyleemphasis{reserved})
&\begin{itemize}
\item {} 
\end{itemize}
&&\\
\hline
\sphinxcode{\sphinxupquote{msgBitmap}}
&
\sphinxcode{\sphinxupquote{FF}}
&
\sphinxcode{\sphinxupquote{D4}}
&
\sphinxcode{\sphinxupquote{04}}
\\
\hline
\end{tabular}
\par
\sphinxattableend\end{savenotes}


\section{OTA API Integration}
\label{\detokenize{otaapi:ota-api-integration}}\label{\detokenize{otaapi::doc}}

\subsection{OTA API}
\label{\detokenize{otaapi:ota-api}}
The over-the-air interface uses \sphinxstylestrong{messages} as the basis for communication.
Messages are retrieved and submitted using Inmarsat’s \sphinxhref{https://developer.inmarsat.com/content/isatdatapro-messaging-api}{IDP Messaging API}.

The key Messaging API operations used are:
\begin{itemize}
\item {} 
\sphinxcode{\sphinxupquote{get\_return\_messages}} is used for retrieving periodic \sphinxstylestrong{reports}, \sphinxstylestrong{alerts} and parameter/data read responses.  See {\hyperref[\detokenize{otaapi:from-mobile-messages}]{\sphinxcrossref{\DUrole{std,std-ref}{From-Mobile Messages}}}}.

\item {} 
\sphinxcode{\sphinxupquote{submit\_messages}} is used for configuration, and on-demand reading/writing of \sphinxstylestrong{parameters} or raw Modbus data.  See {\hyperref[\detokenize{otaapi:to-mobile-messages}]{\sphinxcrossref{\DUrole{std,std-ref}{To-Mobile Messages}}}}.

\end{itemize}


\subsubsection{Message Structure}
\label{\detokenize{otaapi:message-structure}}
Each message is structured as a set of data \sphinxstylestrong{fields} that encapsulate different types of data.

\sphinxstylestrong{Return Messages} (aka Mobile-Originated, From-Mobile) are structured as follows:

\fvset{hllines={, ,}}%
\begin{sphinxVerbatim}[commandchars=\\\{\}]
\PYG{n}{ReturnMessage}\PYG{p}{\PYGZob{}}
  \PYG{n}{Fields}\PYG{p}{[}
    \PYG{n}{ID}\PYG{p}{,}
        \PYG{n}{MessageUTC}\PYG{p}{,}
        \PYG{n}{MobileID}\PYG{p}{,}
        \PYG{n}{OTAMessageSize}\PYG{p}{,}
        \PYG{n}{RawPayload}\PYG{p}{,}
        \PYG{n}{ReceiveUTC}\PYG{p}{,}
        \PYG{n}{RegionName}\PYG{p}{,}
        \PYG{n}{SIN}\PYG{p}{,}
        \PYG{n}{Payload}\PYG{p}{[}
          \PYG{n}{isForward}\PYG{p}{,}
          \PYG{n}{SIN}\PYG{p}{,}
          \PYG{n}{MIN}\PYG{p}{,}
          \PYG{n}{Name}\PYG{p}{,}
          \PYG{n}{Fields}\PYG{p}{[}
                \PYG{p}{\PYGZob{}}
                  \PYG{n}{Name}\PYG{p}{,}
                  \PYG{n}{Type}\PYG{p}{,}
                  \PYG{n}{Value}\PYG{p}{,}
                  \PYG{n}{Elements}\PYG{p}{[}
                        \PYG{p}{\PYGZob{}}
                          \PYG{n}{Index}\PYG{p}{,}
                          \PYG{n}{Fields}\PYG{p}{[}
                                \PYG{p}{\PYGZob{}}
                                  \PYG{n}{Name}\PYG{p}{,}
                                  \PYG{n}{Type}\PYG{p}{,}
                                  \PYG{n}{Value}
                                \PYG{p}{\PYGZcb{}}
                          \PYG{p}{]}
                        \PYG{p}{\PYGZcb{}}
                  \PYG{p}{]}
                \PYG{p}{\PYGZcb{}}
          \PYG{p}{]}
        \PYG{p}{]}
  \PYG{p}{]}
\PYG{p}{\PYGZcb{}}
\end{sphinxVerbatim}

\sphinxstylestrong{Forward Messages} (aka Mobile-Terminated, To-Mobile) are structured as follows:

\fvset{hllines={, ,}}%
\begin{sphinxVerbatim}[commandchars=\\\{\}]
\PYG{n}{ForwardMessage}\PYG{p}{\PYGZob{}}
  \PYG{n}{DestinationID}\PYG{p}{,}
  \PYG{n}{RawPayload}\PYG{p}{,}
  \PYG{n}{UserMessageID}\PYG{p}{,}
  \PYG{n}{Payload}\PYG{p}{[}
        \PYG{n}{isForward}\PYG{p}{,}
        \PYG{n}{SIN}\PYG{p}{,}
        \PYG{n}{MIN}\PYG{p}{,}
        \PYG{n}{Name}\PYG{p}{,}
        \PYG{n}{Fields}\PYG{p}{[}
          \PYG{p}{\PYGZob{}}
                \PYG{n}{Name}\PYG{p}{,}
                \PYG{n}{Type}\PYG{p}{,}
                \PYG{n}{Value}\PYG{p}{,}
                \PYG{n}{Elements}\PYG{p}{[}
                  \PYG{p}{\PYGZob{}}
                        \PYG{n}{Index}\PYG{p}{,}
                        \PYG{n}{Fields}\PYG{p}{[}
                          \PYG{p}{\PYGZob{}}
                            \PYG{n}{Name}\PYG{p}{,}
                                \PYG{n}{Type}\PYG{p}{,}
                                \PYG{n}{Value}
                          \PYG{p}{\PYGZcb{}}
                        \PYG{p}{]}
                  \PYG{p}{\PYGZcb{}}
                \PYG{p}{]}
          \PYG{p}{\PYGZcb{}}
        \PYG{p}{]}
  \PYG{p}{]}
\PYG{p}{\PYGZcb{}}
\end{sphinxVerbatim}

Encoding of \sphinxstylestrong{Fields} used by the Modbus service \sphinxstylestrong{messages} include:
\begin{itemize}
\item {} 
\sphinxcode{\sphinxupquote{boolean}} for values held in Coils or Digital Inputs

\item {} 
\sphinxcode{\sphinxupquote{unsignedInt}} is used for 16-bit unsigned integers
\begin{quote}

\begin{sphinxadmonition}{note}{Note:}
The Lua Service Framework used on IDP-series terminals does not support unsigned 32-bit integers, so the Modbus service also does not support those.
\end{sphinxadmonition}
\end{quote}

\item {} 
\sphinxcode{\sphinxupquote{signedInt}} is used for both 16-bit and 32-bit signed integers

\item {} 
\sphinxcode{\sphinxupquote{string}} is used to transport 32-bit floating point numbers and ASCII data
\begin{quote}

\begin{sphinxadmonition}{note}{Note:}
The Lua Service Framework does not support floating point data types natively as message fields.
Modbus service converts IEEE floating point to a string value of the decimal number.
\end{sphinxadmonition}
\end{quote}

\item {} 
\sphinxcode{\sphinxupquote{data}} is used to transport \sphinxstylestrong{raw} byte arrays

\item {} 
\sphinxcode{\sphinxupquote{array}} is used to encapsulate \sphinxstylestrong{parameters}

\item {} 
\sphinxcode{\sphinxupquote{enum}} is used by some configuration messages

\end{itemize}


\paragraph{Example}
\label{\detokenize{otaapi:example}}
An \sphinxstylestrong{report} with a single \sphinxstylestrong{parameter} \sphinxcode{\sphinxupquote{Payload}} within a \sphinxcode{\sphinxupquote{ReturnMessage}} for {\hyperref[\detokenize{otaapi:periodicreport}]{\sphinxcrossref{\DUrole{std,std-ref}{periodicReport (MIN 100)}}}} is:

\fvset{hllines={, ,}}%
\begin{sphinxVerbatim}[commandchars=\\\{\}]
\PYG{n}{Payload}\PYG{p}{[}
  \PYG{n}{isForward}\PYG{p}{:} \PYG{n}{false}\PYG{p}{,}
  \PYG{n}{SIN}\PYG{p}{:} \PYG{l+m+mi}{200}\PYG{p}{,}
  \PYG{n}{MIN}\PYG{p}{:} \PYG{l+m+mi}{100}\PYG{p}{,}
  \PYG{n}{Name}\PYG{p}{:} \PYG{l+s+s2}{\PYGZdq{}}\PYG{l+s+s2}{periodicReport}\PYG{l+s+s2}{\PYGZdq{}}\PYG{p}{,}
  \PYG{n}{Fields}\PYG{p}{:} \PYG{p}{[}
        \PYG{p}{\PYGZob{}}
          \PYG{n}{Name}\PYG{p}{:} \PYG{l+s+s2}{\PYGZdq{}}\PYG{l+s+s2}{reportId}\PYG{l+s+s2}{\PYGZdq{}}\PYG{p}{,}
          \PYG{n}{Type}\PYG{p}{:} \PYG{l+s+s2}{\PYGZdq{}}\PYG{l+s+s2}{unsignedInt}\PYG{l+s+s2}{\PYGZdq{}}\PYG{p}{,}
          \PYG{n}{Value}\PYG{p}{:} \PYG{l+m+mi}{1}
        \PYG{p}{\PYGZcb{}}\PYG{p}{,}
        \PYG{p}{\PYGZob{}}
          \PYG{n}{Name}\PYG{p}{:} \PYG{l+s+s2}{\PYGZdq{}}\PYG{l+s+s2}{parameters}\PYG{l+s+s2}{\PYGZdq{}}\PYG{p}{,}
          \PYG{n}{Type}\PYG{p}{:} \PYG{l+s+s2}{\PYGZdq{}}\PYG{l+s+s2}{array}\PYG{l+s+s2}{\PYGZdq{}}\PYG{p}{,}
          \PYG{n}{Elements}\PYG{p}{:} \PYG{p}{[}
                \PYG{p}{\PYGZob{}}
                  \PYG{n}{Index}\PYG{p}{:} \PYG{l+m+mi}{0}\PYG{p}{,}
                  \PYG{n}{Fields}\PYG{p}{:} \PYG{p}{[}
                        \PYG{p}{\PYGZob{}}
                          \PYG{n}{Name}\PYG{p}{:} \PYG{l+s+s2}{\PYGZdq{}}\PYG{l+s+s2}{paramId}\PYG{l+s+s2}{\PYGZdq{}}\PYG{p}{,}
                          \PYG{n}{Type}\PYG{p}{:} \PYG{l+s+s2}{\PYGZdq{}}\PYG{l+s+s2}{unsignedInt}\PYG{l+s+s2}{\PYGZdq{}}\PYG{p}{,}
                          \PYG{n}{Value}\PYG{p}{:} \PYG{l+m+mi}{1}
                        \PYG{p}{\PYGZcb{}}\PYG{p}{,}
                        \PYG{p}{\PYGZob{}}
                          \PYG{n}{Name}\PYG{p}{:} \PYG{l+s+s2}{\PYGZdq{}}\PYG{l+s+s2}{value}\PYG{l+s+s2}{\PYGZdq{}}\PYG{p}{,}
                          \PYG{n}{Type}\PYG{p}{:} \PYG{l+s+s2}{\PYGZdq{}}\PYG{l+s+s2}{unsignedInt}\PYG{l+s+s2}{\PYGZdq{}}\PYG{p}{,}
                          \PYG{n}{Value}\PYG{p}{:} \PYG{l+m+mi}{300}
                        \PYG{p}{\PYGZcb{}}
                  \PYG{p}{]}
                \PYG{p}{\PYGZcb{}}
          \PYG{p}{]}
        \PYG{p}{\PYGZcb{}}
  \PYG{p}{]}
\PYG{p}{]}
\end{sphinxVerbatim}


\bigskip\hrule\bigskip


\begin{sphinxadmonition}{note}{Note:}
The following sections assumes the reader is familiar with the ORBCOMM Lua Terminal API documentation and how to read the tables.
\end{sphinxadmonition}

\begin{sphinxadmonition}{warning}{Warning:}
This section is under construction!
\end{sphinxadmonition}


\subsubsection{To-Mobile Messages}
\label{\detokenize{otaapi:to-mobile-messages}}\label{\detokenize{otaapi:id1}}

\begin{savenotes}\sphinxattablestart
\centering
\sphinxcapstartof{table}
\sphinxcaption{To-Mobile Messages}\label{\detokenize{otaapi:id3}}
\sphinxaftercaption
\begin{tabular}[t]{|\X{10}{105}|\X{25}{105}|\X{70}{105}|}
\hline
\sphinxstyletheadfamily 
MIN
&\sphinxstyletheadfamily 
Name
&\sphinxstyletheadfamily 
Description
\\
\hline
10
&
{\hyperref[\detokenize{otaapi:setmodbusconfig}]{\sphinxcrossref{\DUrole{std,std-ref}{setModbusConfig}}}}
&
Configure \sphinxstylestrong{ports}, \sphinxstylestrong{devices}, \sphinxstylestrong{parameters} and \sphinxstylestrong{properties}
\\
\hline
11
&
{\hyperref[\detokenize{otaapi:getmodbusconfig}]{\sphinxcrossref{\DUrole{std,std-ref}{getModbusConfig}}}}
&
Get configuration of \sphinxstylestrong{ports}, \sphinxstylestrong{devices}, \sphinxstylestrong{parameters} and \sphinxstylestrong{properties}
\\
\hline
12
&
{\hyperref[\detokenize{otaapi:delmodbusconfig}]{\sphinxcrossref{\DUrole{std,std-ref}{delModbusConfig}}}}
&
Delete configuration(s)
\\
\hline
13
&
{\hyperref[\detokenize{otaapi:setreportconfig}]{\sphinxcrossref{\DUrole{std,std-ref}{setReportConfig}}}}
&
Configure periodic \sphinxstylestrong{reports}
\\
\hline
14
&
{\hyperref[\detokenize{otaapi:getreportconfig}]{\sphinxcrossref{\DUrole{std,std-ref}{getReportConfig}}}}
&
Get configuration of periodic \sphinxstylestrong{reports}
\\
\hline
15
&
{\hyperref[\detokenize{otaapi:delreportconfig}]{\sphinxcrossref{\DUrole{std,std-ref}{delReportConfig}}}}
&
Delete periodic \sphinxstylestrong{reports}
\\
\hline
16
&
{\hyperref[\detokenize{otaapi:setalertconfig}]{\sphinxcrossref{\DUrole{std,std-ref}{setAlertConfig}}}}
&
Configure \sphinxstylestrong{alerts}
\\
\hline
17
&
{\hyperref[\detokenize{otaapi:getalertconfig}]{\sphinxcrossref{\DUrole{std,std-ref}{getAlertConfig}}}}
&
Get configuration of \sphinxstylestrong{alerts}
\\
\hline
18
&
{\hyperref[\detokenize{otaapi:delalertconfig}]{\sphinxcrossref{\DUrole{std,std-ref}{delAlertConfig}}}}
&
Delete configuration of \sphinxstylestrong{alerts}
\\
\hline
19
&
{\hyperref[\detokenize{otaapi:readparam}]{\sphinxcrossref{\DUrole{std,std-ref}{readParam}}}}
&
Read \sphinxstylestrong{parameters} data values immediately.
\\
\hline
20
&
{\hyperref[\detokenize{otaapi:writeparam}]{\sphinxcrossref{\DUrole{std,std-ref}{writeParam}}}}
&
Write \sphinxstylestrong{parameters} data values immediately.
\\
\hline
21
&
{\hyperref[\detokenize{otaapi:readdata}]{\sphinxcrossref{\DUrole{std,std-ref}{readData}}}}
&
Read a Modbus register directly.
\\
\hline
22
&
{\hyperref[\detokenize{otaapi:writedata}]{\sphinxcrossref{\DUrole{std,std-ref}{writeData}}}}
&
Write a Modbus register directly.
\\
\hline
23
&
{\hyperref[\detokenize{otaapi:setrefposition}]{\sphinxcrossref{\DUrole{std,std-ref}{setRefPosition}}}}
&
Sets the current location as the \sphinxstyleemphasis{geolocation} reference, and acknowledges a \sphinxstylestrong{positionChangeAlert}
\\
\hline
\end{tabular}
\par
\sphinxattableend\end{savenotes}


\bigskip\hrule\bigskip



\paragraph{setModbusConfig (MIN 10)}
\label{\detokenize{otaapi:setmodbusconfig-min-10}}\label{\detokenize{otaapi:setmodbusconfig}}
Configures Modbus \sphinxstylestrong{ports}, \sphinxstylestrong{devices}, \sphinxstylestrong{parameters} (and optionally \sphinxstylestrong{properties}).


\begin{savenotes}\sphinxattablestart
\centering
\sphinxcapstartof{table}
\sphinxcaption{setModbusConfig}\label{\detokenize{otaapi:id4}}
\sphinxaftercaption
\begin{tabular}[t]{|\X{25}{100}|\X{10}{100}|\X{5}{100}|\X{5}{100}|\X{5}{100}|\X{50}{100}|}
\hline
\sphinxstyletheadfamily 
Field Name
&\sphinxstyletheadfamily 
Type
&\sphinxstyletheadfamily 
Optional
&\sphinxstyletheadfamily 
Fixed
&\sphinxstyletheadfamily 
Size
&\sphinxstyletheadfamily 
Comments
\\
\hline
ports
&
array
&
Y
&
N
&
3
&
A serial port used to connect to one or more Modbus devices.
\\
\hline\begin{itemize}
\item {} 
port

\end{itemize}
&
enum
&
N
&
N/A
&
2
&
\begin{DUlineblock}{0em}
\item[] Name of the port
\item[] \sphinxcode{\sphinxupquote{rs232}}
\item[] \sphinxcode{\sphinxupquote{rs485}}
\item[] \sphinxcode{\sphinxupquote{rs232aux}}
\end{DUlineblock}
\\
\hline\begin{itemize}
\item {} 
baudRate

\end{itemize}
&
enum
&
N
&
N/A
&
3
&
\begin{DUlineblock}{0em}
\item[] Speed of serial communications.
\item[] \sphinxcode{\sphinxupquote{1200}}, \sphinxcode{\sphinxupquote{2400}}, \sphinxcode{\sphinxupquote{4800}}, \sphinxcode{\sphinxupquote{9600}}, \sphinxcode{\sphinxupquote{19200}}, \sphinxcode{\sphinxupquote{38400}}, \sphinxcode{\sphinxupquote{57600}}, \sphinxcode{\sphinxupquote{115200}}
\end{DUlineblock}
\\
\hline\begin{itemize}
\item {} 
parity

\end{itemize}
&
enum
&
N
&
N/A
&
2
&
\begin{DUlineblock}{0em}
\item[] \sphinxcode{\sphinxupquote{none}}
\item[] \sphinxcode{\sphinxupquote{even}}
\item[] \sphinxcode{\sphinxupquote{odd}}
\end{DUlineblock}
\\
\hline\begin{itemize}
\item {} 
mode

\end{itemize}
&
enum
&
N
&
N/A
&
1
&
\sphinxcode{\sphinxupquote{rtu}} or \sphinxcode{\sphinxupquote{ascii}}
\\
\hline
devices
&
array
&
Y
&
N
&
100
&
Configuration of Modbus slave devices.
\\
\hline\begin{itemize}
\item {} 
deviceId

\end{itemize}
&
unsignedInt
&
N
&
N/A
&
8
&
Unique identifier of the device in the Modbus service.
\\
\hline\begin{itemize}
\item {} 
networkId

\end{itemize}
&
unsignedInt
&
N
&
N/A
&
8
&
Modbus Slave address on the physical network.
\\
\hline\begin{itemize}
\item {} 
port

\end{itemize}
&
enum
&
N
&
N/A
&
2
&
Serial port on which the device is attached.
\\
\hline\begin{itemize}
\item {} 
pollInterval

\end{itemize}
&
unsignedInt
&
N
&
N/A
&
17
&
The interval (seconds) at which the device registers are polled.
\\
\hline\begin{itemize}
\item {} 
byteOrder

\end{itemize}
&
enum
&
Y
&
N/A
&
1
&
\sphinxcode{\sphinxupquote{msb}} or \sphinxcode{\sphinxupquote{lsb}}
\\
\hline\begin{itemize}
\item {} 
wordOrder

\end{itemize}
&
enum
&
Y
&
N/A
&
1
&
\sphinxcode{\sphinxupquote{msw}} or \sphinxcode{\sphinxupquote{lsw}}
\\
\hline\begin{itemize}
\item {} 
serialRxTimeout

\end{itemize}
&
unsignedInt
&
Y
&
N/A
&
15
&
Read timeout (milliseconds) waiting on next byte to be received.
\\
\hline\begin{itemize}
\item {} 
serialTxTimeout

\end{itemize}
&
unsignedInt
&
Y
&
N/A
&
15
&
Read timeout (milliseconds) waiting on last transmitted byte to be confirmed.
\\
\hline\begin{itemize}
\item {} 
plcBaseAddress

\end{itemize}
&
enum
&
Y
&
N/A
&
1
&
1 = Use PLC addressing (non zero-based)
\\
\hline\begin{itemize}
\item {} 
retries

\end{itemize}
&
unsignedInt
&
Y
&
N/A
&
4
&
The number of serial retries to attempt before raising a communication error.
\\
\hline
parameters
&
array
&
Y
&
N
&
300
&
Configuration of parameters used by the service.
\\
\hline\begin{itemize}
\item {} 
paramId

\end{itemize}
&
unsignedInt
&
N
&
N/A
&
10
&
Unique identifier of the parameter used for reporting and analytics.
\\
\hline\begin{itemize}
\item {} 
deviceId

\end{itemize}
&
unsignedInt
&
N
&
N/A
&
8
&
Unique identifier of the device in the Modbus service.
\\
\hline\begin{itemize}
\item {} 
registerType

\end{itemize}
&
enum
&
N
&
N/A
&
2
&
\begin{DUlineblock}{0em}
\item[] \sphinxcode{\sphinxupquote{coil}} - discrete output coil
\item[] \sphinxcode{\sphinxupquote{input}} - discrete digital input
\item[] \sphinxcode{\sphinxupquote{analog}} - (analog) input register
\item[] \sphinxcode{\sphinxupquote{holding}} - holding register
\end{DUlineblock}
\\
\hline\begin{itemize}
\item {} 
encoding

\end{itemize}
&
enum
&
N
&
N/A
&
4
&
\sphinxcode{\sphinxupquote{boolean}}, \sphinxcode{\sphinxupquote{int8}}, \sphinxcode{\sphinxupquote{uint8}}, \sphinxcode{\sphinxupquote{int16}}, \sphinxcode{\sphinxupquote{uint16}}, \sphinxcode{\sphinxupquote{int32}}, \sphinxcode{\sphinxupquote{float32}}, \sphinxcode{\sphinxupquote{string}}, \sphinxcode{\sphinxupquote{base64string}}, \sphinxcode{\sphinxupquote{raw}}
\\
\hline\begin{itemize}
\item {} 
address

\end{itemize}
&
unsignedInt
&
N
&
N/A
&
31
&
Starting Modbus register address of the data
\\
\hline\begin{itemize}
\item {} 
length

\end{itemize}
&
unsignedInt
&
Y
&
N/A
&
12
&
Number of registers used to contain the data (including the starting address)
\\
\hline\begin{itemize}
\item {} 
mult

\end{itemize}
&
unsignedInt
&
Y
&
N/A
&
7
&
A multiplier applied to the data value prior to reporting or analytics.
\\
\hline
properties
&
array
&
Y
&
N/A
&
245
&
Mapping of parameters to properties.
\\
\hline\begin{itemize}
\item {} 
paramId

\end{itemize}
&
unsignedInt
&
N
&
N/A
&
10
&
Unique identifier of the parameter used for reporting and analytics.
\\
\hline\begin{itemize}
\item {} 
pin

\end{itemize}
&
unsignedInt
&
N
&
N/A
&
8
&
The property identification number (within Modbus service) to hold the parameter value.
\\
\hline
\end{tabular}
\par
\sphinxattableend\end{savenotes}


\paragraph{getModbusConfig (MIN 11)}
\label{\detokenize{otaapi:getmodbusconfig-min-11}}\label{\detokenize{otaapi:getmodbusconfig}}
Retrieves select Modbus configuration(s).


\begin{savenotes}\sphinxattablestart
\centering
\sphinxcapstartof{table}
\sphinxcaption{getModbusConfig}\label{\detokenize{otaapi:id5}}
\sphinxaftercaption
\begin{tabular}[t]{|\X{25}{100}|\X{10}{100}|\X{5}{100}|\X{5}{100}|\X{5}{100}|\X{50}{100}|}
\hline
\sphinxstyletheadfamily 
Field Name
&\sphinxstyletheadfamily 
Type
&\sphinxstyletheadfamily 
Optional
&\sphinxstyletheadfamily 
Fixed
&\sphinxstyletheadfamily 
Size
&\sphinxstyletheadfamily 
Comments
\\
\hline
port
&
enum
&
Y
&
Y
&
3
&
\begin{DUlineblock}{0em}
\item[] If included, specifies the \sphinxstylestrong{port} configuration being queried. If \sphinxstyleemphasis{not} included, all ports configurations will be returned.
\item[] \sphinxcode{\sphinxupquote{rs232}}
\item[] \sphinxcode{\sphinxupquote{rs485}}
\item[] \sphinxcode{\sphinxupquote{rs232aux}}
\end{DUlineblock}
\\
\hline
deviceId
&
unsignedInt
&
Y
&
Y
&
8
&
If included, specifies the \sphinxstylestrong{device} configuration being queried.  If \sphinxstyleemphasis{not} included, all device configurations will be returned.
\\
\hline
paramId
&
unsignedInt
&
Y
&
Y
&
10
&
If included, specifies the \sphinxstylestrong{parameter} configuration being queried.  If \sphinxstyleemphasis{not} included, all parameters configurations will be returned.
\\
\hline
properties
&
boolean
&
Y
&
Y
&
1
&
If included, \sphinxstylestrong{properties} configured will be returned.
\\
\hline
\end{tabular}
\par
\sphinxattableend\end{savenotes}


\paragraph{delModbusConfig (MIN 12)}
\label{\detokenize{otaapi:delmodbusconfig-min-12}}\label{\detokenize{otaapi:delmodbusconfig}}
Deletes select Modbus configuration(s).


\begin{savenotes}\sphinxattablestart
\centering
\sphinxcapstartof{table}
\sphinxcaption{delModbusConfig}\label{\detokenize{otaapi:id6}}
\sphinxaftercaption
\begin{tabular}[t]{|\X{25}{100}|\X{10}{100}|\X{5}{100}|\X{5}{100}|\X{5}{100}|\X{50}{100}|}
\hline
\sphinxstyletheadfamily 
Field Name
&\sphinxstyletheadfamily 
Type
&\sphinxstyletheadfamily 
Optional
&\sphinxstyletheadfamily 
Fixed
&\sphinxstyletheadfamily 
Size
&\sphinxstyletheadfamily 
Comments
\\
\hline
ports
&
array
&
Y
&
N
&
3
&
If included, configuration for the specified \sphinxstylestrong{ports} will be deleted.  If \sphinxstyleemphasis{not} included, no ports will be deleted.
\\
\hline\begin{itemize}
\item {} 
port

\end{itemize}
&
enum
&
N
&
Y
&
2
&
\begin{DUlineblock}{0em}
\item[] Name of the \sphinxstylestrong{port} configuration to delete.
\item[] \sphinxcode{\sphinxupquote{rs232}}
\item[] \sphinxcode{\sphinxupquote{rs485}}
\item[] \sphinxcode{\sphinxupquote{rs232aux}}
\end{DUlineblock}
\\
\hline
devices
&
array
&
Y
&
N
&
100
&
If included, configuration for the specified \sphinxstylestrong{devices} will be deleted.  If \sphinxstyleemphasis{not} included, no devices will be deleted.
\\
\hline\begin{itemize}
\item {} 
deviceId

\end{itemize}
&
unsignedInt
&
N
&
Y
&
8
&
Unique identifier of the \sphinxstylestrong{device} to delete.
\\
\hline
parameters
&
array
&
Y
&
N
&
300
&
If included, configuration for the specified \sphinxstylestrong{parameters} will be deleted.  If \sphinxstyleemphasis{not} included, no parameters will be deleted.
\\
\hline\begin{itemize}
\item {} 
paramId

\end{itemize}
&
unsignedInt
&
N
&
N/A
&
10
&
Unique identifier of the \sphinxstylestrong{parameter} to delete.
\\
\hline
properties
&
array
&
Y
&
N
&
245
&
If included, configuration for the specified \sphinxstylestrong{properties} will be deleted.  If \sphinxstyleemphasis{not} included, no properties will be deleted.
\\
\hline\begin{itemize}
\item {} 
property

\end{itemize}
&
unsignedInt
&
N
&
N/A
&
10
&
The property identification number (within Modbus service) to hold the parameter value.
\\
\hline
\end{tabular}
\par
\sphinxattableend\end{savenotes}


\paragraph{setReportConfig (MIN 13)}
\label{\detokenize{otaapi:setreportconfig-min-13}}\label{\detokenize{otaapi:setreportconfig}}
Configures periodic \sphinxstylestrong{reports}.


\begin{savenotes}\sphinxattablestart
\centering
\sphinxcapstartof{table}
\sphinxcaption{setReportConfig}\label{\detokenize{otaapi:id7}}
\sphinxaftercaption
\begin{tabular}[t]{|\X{25}{100}|\X{10}{100}|\X{5}{100}|\X{5}{100}|\X{5}{100}|\X{50}{100}|}
\hline
\sphinxstyletheadfamily 
Field Name
&\sphinxstyletheadfamily 
Type
&\sphinxstyletheadfamily 
Optional
&\sphinxstyletheadfamily 
Fixed
&\sphinxstyletheadfamily 
Size
&\sphinxstyletheadfamily 
Comments
\\
\hline
reports
&
array
&
N
&
N
&
100
&
The list of periodic \sphinxstylestrong{reports} to configure.
\\
\hline\begin{itemize}
\item {} 
reportId

\end{itemize}
&
unsignedInt
&
N
&
Y
&
31
&
Unique identifier of the \sphinxstylestrong{report} associated with particular parameter values.
\\
\hline\begin{itemize}
\item {} 
paramIds

\end{itemize}
&
string
&
N
&
N
&
255
&
Comma-separated list of \sphinxstylestrong{parameter} ids to be included in the \sphinxstylestrong{report}.
\\
\hline\begin{itemize}
\item {} 
interval

\end{itemize}
&
unsignedInt
&
N
&
Y
&
20
&
The periodic \sphinxstylestrong{report} interval, in \sphinxstyleemphasis{seconds}.  The first \sphinxstylestrong{report} will be sent as soon as the new configuration is applied.
\\
\hline\begin{itemize}
\item {} 
timeOfDay

\end{itemize}
&
unsignedInt
&
Y
&
Y
&
18
&
\sphinxstyleemphasis{Optional} time of day, in seconds (0..86400) to create a snapshot that will be reported within the \sphinxstylestrong{timeOfDayWindow}.
\\
\hline\begin{itemize}
\item {} 
timeOfDayWindow

\end{itemize}
&
unsignedInt
&
Y
&
Y
&
13
&
\sphinxstyleemphasis{Must be included} if using \sphinxstylestrong{timeOfDay}.  Window, in seconds, within which the \sphinxstylestrong{report} will be sent offset relative to the \sphinxstylestrong{timeOfDay} snapshot.
\\
\hline\begin{itemize}
\item {} 
msgBitMap

\end{itemize}
&
string
&
Y
&
Y
&
2
&
ASCII string of \sphinxstyleemphasis{metadata} fields to include, overriding \sphinxstylestrong{defaultMsgBitmap} value.  See {\hyperref[\detokenize{configuration:msgbitmap}]{\sphinxcrossref{\DUrole{std,std-ref}{Optional Metadata Fields Configuration (MsgBitmap, defaultMsgBitmap)}}}}.
\\
\hline
\end{tabular}
\par
\sphinxattableend\end{savenotes}


\paragraph{getReportConfig (MIN 14)}
\label{\detokenize{otaapi:getreportconfig-min-14}}\label{\detokenize{otaapi:getreportconfig}}
Retrieves select Report configuration(s).


\begin{savenotes}\sphinxattablestart
\centering
\sphinxcapstartof{table}
\sphinxcaption{getReportConfig}\label{\detokenize{otaapi:id8}}
\sphinxaftercaption
\begin{tabular}[t]{|\X{25}{100}|\X{10}{100}|\X{5}{100}|\X{5}{100}|\X{5}{100}|\X{50}{100}|}
\hline
\sphinxstyletheadfamily 
Field Name
&\sphinxstyletheadfamily 
Type
&\sphinxstyletheadfamily 
Optional
&\sphinxstyletheadfamily 
Fixed
&\sphinxstyletheadfamily 
Size
&\sphinxstyletheadfamily 
Comments
\\
\hline
reportId
&
unsignedInt
&
Y
&
Y
&
31
&
If included, specifies the \sphinxstylestrong{report} configuration being queried. If \sphinxstyleemphasis{not} included, all reports configurations will be returned.
\\
\hline
\end{tabular}
\par
\sphinxattableend\end{savenotes}


\paragraph{delReportConfig (MIN 15)}
\label{\detokenize{otaapi:delreportconfig-min-15}}\label{\detokenize{otaapi:delreportconfig}}
Deletes select Report configuration(s).


\begin{savenotes}\sphinxattablestart
\centering
\sphinxcapstartof{table}
\sphinxcaption{delReportConfig}\label{\detokenize{otaapi:id9}}
\sphinxaftercaption
\begin{tabular}[t]{|\X{25}{100}|\X{10}{100}|\X{5}{100}|\X{5}{100}|\X{5}{100}|\X{50}{100}|}
\hline
\sphinxstyletheadfamily 
Field Name
&\sphinxstyletheadfamily 
Type
&\sphinxstyletheadfamily 
Optional
&\sphinxstyletheadfamily 
Fixed
&\sphinxstyletheadfamily 
Size
&\sphinxstyletheadfamily 
Comments
\\
\hline
reportId
&
unsignedInt
&
Y
&
Y
&
31
&
If included, specifies the \sphinxstylestrong{report} configuration being deleted.

\begin{sphinxadmonition}{warning}{Warning:}
If \sphinxstyleemphasis{not} included, all reports configurations will be deleted.
\end{sphinxadmonition}
\\
\hline
\end{tabular}
\par
\sphinxattableend\end{savenotes}


\paragraph{setAlertConfig (MIN 16)}
\label{\detokenize{otaapi:setalertconfig-min-16}}\label{\detokenize{otaapi:setalertconfig}}
Configures \sphinxstylestrong{alerts}.


\begin{savenotes}\sphinxattablestart
\centering
\sphinxcapstartof{table}
\sphinxcaption{setAlertConfig}\label{\detokenize{otaapi:id10}}
\sphinxaftercaption
\begin{tabular}[t]{|\X{25}{100}|\X{10}{100}|\X{5}{100}|\X{5}{100}|\X{5}{100}|\X{50}{100}|}
\hline
\sphinxstyletheadfamily 
Field Name
&\sphinxstyletheadfamily 
Type
&\sphinxstyletheadfamily 
Optional
&\sphinxstyletheadfamily 
Fixed
&\sphinxstyletheadfamily 
Size
&\sphinxstyletheadfamily 
Comments
\\
\hline
alerts
&
array
&
N
&
N
&
100
&
The list of \sphinxstylestrong{alerts} to configure.
\\
\hline\begin{itemize}
\item {} 
alertId

\end{itemize}
&
unsignedInt
&
N
&
Y
&
31
&
Unique identifier of the \sphinxstylestrong{alert} associated with a particular parameter trigger.
\\
\hline\begin{itemize}
\item {} 
paramId

\end{itemize}
&
unsignedInt
&
N
&
Y
&
10
&
The ID of the trigger \sphinxstylestrong{parameter} on which the analytics is run.
\\
\hline\begin{itemize}
\item {} 
paramIds

\end{itemize}
&
string
&
Y
&
N
&
255
&
Comma-separated list of \sphinxstylestrong{parameter} values to be included in the \sphinxstylestrong{alert}, in addition to the trigger parameter.
\\
\hline\begin{itemize}
\item {} 
minON

\end{itemize}
&
int
&
Y
&
Y
&
32
&
The lower threshold against which the parameter is tested.  If the value drops below this, an \sphinxstylestrong{alertON} is sent.  The parameter stays in the alert state until the value rises above \sphinxstylestrong{minOFF}.
\\
\hline\begin{itemize}
\item {} 
minOFF

\end{itemize}
&
int
&
Y
&
Y
&
32
&
The normal low threshold against which the parameter is tested.  If the parameter is in the alert state and rises above this value, the alert state is off and an \sphinxstylestrong{alertOFF} is sent.

If \sphinxstyleemphasis{not} specified, \sphinxstylestrong{minON} will be used as the return to normal trigger.
\\
\hline\begin{itemize}
\item {} 
maxON

\end{itemize}
&
int
&
Y
&
Y
&
32
&
The upper threshold against which the parameter is tested.  If the value rises above this, an \sphinxstylestrong{alertON} is sent.  The parameter stays in the alert state until the value drops below \sphinxstylestrong{maxOFF}.
\\
\hline\begin{itemize}
\item {} 
maxOFF

\end{itemize}
&
int
&
Y
&
Y
&
32
&
The normal high threshold against which the parameter is tested.  If the parameter is in the alert state and drops below this value, the alert state is off and an \sphinxstylestrong{alertOFF} is sent.

If \sphinxstyleemphasis{not} specified, \sphinxstylestrong{maxON} will be used as the return to normal trigger.
\\
\hline\begin{itemize}
\item {} 
change

\end{itemize}
&
unsignedInt
&
Y
&
Y
&
31
&
The absolute value difference from the last reported value, which will trigger a \sphinxstylestrong{alertChange} to be sent.
\\
\hline\begin{itemize}
\item {} 
msgBitMap

\end{itemize}
&
string
&
Y
&
Y
&
2
&
ASCII string of \sphinxstyleemphasis{metadata} fields to include, overriding \sphinxstylestrong{defaultMsgBitmap} value.  See {\hyperref[\detokenize{configuration:msgbitmap}]{\sphinxcrossref{\DUrole{std,std-ref}{Optional Metadata Fields Configuration (MsgBitmap, defaultMsgBitmap)}}}}.
\\
\hline
\end{tabular}
\par
\sphinxattableend\end{savenotes}


\paragraph{getAlertConfig (MIN 17)}
\label{\detokenize{otaapi:getalertconfig-min-17}}\label{\detokenize{otaapi:getalertconfig}}
Retrieves select Alert configuration(s).


\begin{savenotes}\sphinxattablestart
\centering
\sphinxcapstartof{table}
\sphinxcaption{getAlertConfig}\label{\detokenize{otaapi:id11}}
\sphinxaftercaption
\begin{tabular}[t]{|\X{25}{100}|\X{10}{100}|\X{5}{100}|\X{5}{100}|\X{5}{100}|\X{50}{100}|}
\hline
\sphinxstyletheadfamily 
Field Name
&\sphinxstyletheadfamily 
Type
&\sphinxstyletheadfamily 
Optional
&\sphinxstyletheadfamily 
Fixed
&\sphinxstyletheadfamily 
Size
&\sphinxstyletheadfamily 
Comments
\\
\hline
alertId
&
unsignedInt
&
Y
&
Y
&
31
&
If included, specifies the \sphinxstylestrong{alert} configuration being queried. If \sphinxstyleemphasis{not} included, all alerts configurations will be returned.
\\
\hline
\end{tabular}
\par
\sphinxattableend\end{savenotes}


\paragraph{delAlertConfig (MIN 18)}
\label{\detokenize{otaapi:delalertconfig-min-18}}\label{\detokenize{otaapi:delalertconfig}}
Deletes select Alert configuration(s).


\begin{savenotes}\sphinxattablestart
\centering
\sphinxcapstartof{table}
\sphinxcaption{delAlertConfig}\label{\detokenize{otaapi:id12}}
\sphinxaftercaption
\begin{tabular}[t]{|\X{25}{100}|\X{10}{100}|\X{5}{100}|\X{5}{100}|\X{5}{100}|\X{50}{100}|}
\hline
\sphinxstyletheadfamily 
Field Name
&\sphinxstyletheadfamily 
Type
&\sphinxstyletheadfamily 
Optional
&\sphinxstyletheadfamily 
Fixed
&\sphinxstyletheadfamily 
Size
&\sphinxstyletheadfamily 
Comments
\\
\hline
alertId
&
unsignedInt
&
Y
&
Y
&
31
&
If included, specifies the \sphinxstylestrong{alert} configuration being deleted.

\begin{sphinxadmonition}{warning}{Warning:}
If \sphinxstyleemphasis{not} included, all alerts configurations will be deleted.
\end{sphinxadmonition}
\\
\hline
\end{tabular}
\par
\sphinxattableend\end{savenotes}


\paragraph{readParam (MIN 19)}
\label{\detokenize{otaapi:readparam-min-19}}\label{\detokenize{otaapi:readparam}}
Reads select \sphinxstylestrong{parameter} values immediately.


\begin{savenotes}\sphinxattablestart
\centering
\sphinxcapstartof{table}
\sphinxcaption{readParam}\label{\detokenize{otaapi:id13}}
\sphinxaftercaption
\begin{tabular}[t]{|\X{25}{100}|\X{10}{100}|\X{5}{100}|\X{5}{100}|\X{5}{100}|\X{50}{100}|}
\hline
\sphinxstyletheadfamily 
Field Name
&\sphinxstyletheadfamily 
Type
&\sphinxstyletheadfamily 
Optional
&\sphinxstyletheadfamily 
Fixed
&\sphinxstyletheadfamily 
Size
&\sphinxstyletheadfamily 
Comments
\\
\hline
parameters
&
array
&
N
&
N
&
100
&
The list of \sphinxstylestrong{parameter} values being queried.
\\
\hline\begin{itemize}
\item {} 
paramId

\end{itemize}
&
unsignedInt
&
N
&
Y
&
10
&
Unique identifier of the \sphinxstylestrong{parameter} to query.
\\
\hline
\end{tabular}
\par
\sphinxattableend\end{savenotes}


\paragraph{writeParam (MIN 20)}
\label{\detokenize{otaapi:writeparam-min-20}}\label{\detokenize{otaapi:writeparam}}
Writes select \sphinxstylestrong{parameter} values immediately.  (Must be \sphinxcode{\sphinxupquote{coil}} or \sphinxcode{\sphinxupquote{holding}} types)


\begin{savenotes}\sphinxattablestart
\centering
\sphinxcapstartof{table}
\sphinxcaption{writeParam}\label{\detokenize{otaapi:id14}}
\sphinxaftercaption
\begin{tabular}[t]{|\X{25}{100}|\X{10}{100}|\X{5}{100}|\X{5}{100}|\X{5}{100}|\X{50}{100}|}
\hline
\sphinxstyletheadfamily 
Field Name
&\sphinxstyletheadfamily 
Type
&\sphinxstyletheadfamily 
Optional
&\sphinxstyletheadfamily 
Fixed
&\sphinxstyletheadfamily 
Size
&\sphinxstyletheadfamily 
Comments
\\
\hline
parameters
&
array
&
N
&
N
&
100
&
The list of \sphinxstylestrong{parameter} values being written.
\\
\hline\begin{itemize}
\item {} 
paramId

\end{itemize}
&
unsignedInt
&
N
&
Y
&
10
&
Unique identifier of the \sphinxstylestrong{parameter} to write.
\\
\hline\begin{itemize}
\item {} 
value

\end{itemize}
&
string
&
N
&
N
&
255
&
The value, in \sphinxstyleemphasis{string} format, to write to the \sphinxstylestrong{parameter}.
\\
\hline
\end{tabular}
\par
\sphinxattableend\end{savenotes}


\paragraph{readData (MIN 21)}
\label{\detokenize{otaapi:readdata-min-21}}\label{\detokenize{otaapi:readdata}}
Reads select blocks of Modbus addresses directly.


\begin{savenotes}\sphinxattablestart
\centering
\sphinxcapstartof{table}
\sphinxcaption{readData}\label{\detokenize{otaapi:id15}}
\sphinxaftercaption
\begin{tabular}[t]{|\X{25}{100}|\X{10}{100}|\X{5}{100}|\X{5}{100}|\X{5}{100}|\X{50}{100}|}
\hline
\sphinxstyletheadfamily 
Field Name
&\sphinxstyletheadfamily 
Type
&\sphinxstyletheadfamily 
Optional
&\sphinxstyletheadfamily 
Fixed
&\sphinxstyletheadfamily 
Size
&\sphinxstyletheadfamily 
Comments
\\
\hline
addresses
&
array
&
N
&
N
&
100
&
The list of Modbus register addresses to be read.
\\
\hline\begin{itemize}
\item {} 
deviceId

\end{itemize}
&
unsignedInt
&
N
&
Y
&
8
&
Unique identifier of the \sphinxstylestrong{device} in the configuration.
\\
\hline\begin{itemize}
\item {} 
registerType

\end{itemize}
&
enum
&
N
&
Y
&
2
&
\begin{DUlineblock}{0em}
\item[] The register type:
\item[] \sphinxcode{\sphinxupquote{coil}}
\item[] \sphinxcode{\sphinxupquote{input}}
\item[] \sphinxcode{\sphinxupquote{inputRegister}}
\item[] \sphinxcode{\sphinxupquote{holdingRegister}}
\end{DUlineblock}
\\
\hline\begin{itemize}
\item {} 
address

\end{itemize}
&
unsignedInt
&
N
&
Y
&
31
&
The starting address to read.  (May be offset if \sphinxstylestrong{device} is configured with \sphinxstylestrong{plcBaseAddress})
\\
\hline\begin{itemize}
\item {} 
encoding

\end{itemize}
&
enum
&
N
&
Y
&
4
&
\begin{DUlineblock}{0em}
\item[] Data conversion to apply to the requested value when responding.
\item[] \sphinxcode{\sphinxupquote{boolean}}
\item[] \sphinxcode{\sphinxupquote{int8}}
\item[] \sphinxcode{\sphinxupquote{uInt8}}
\item[] \sphinxcode{\sphinxupquote{int16}}
\item[] \sphinxcode{\sphinxupquote{uInt16}}
\item[] \sphinxcode{\sphinxupquote{int32}}
\item[] \sphinxcode{\sphinxupquote{float32}}
\item[] \sphinxcode{\sphinxupquote{string}}
\item[] \sphinxcode{\sphinxupquote{base64string}}
\item[] \sphinxcode{\sphinxupquote{raw}}
\end{DUlineblock}
\\
\hline\begin{itemize}
\item {} 
length

\end{itemize}
&
unsignedInt
&
Y
&
Y
&
12
&
The number of registers to read.  This is \sphinxstyleemphasis{implied} for all encoding types except \sphinxcode{\sphinxupquote{string}}, \sphinxcode{\sphinxupquote{base64string}} and \sphinxcode{\sphinxupquote{raw}}
\\
\hline\begin{itemize}
\item {} 
byteOrder

\end{itemize}
&
enum
&
Y
&
Y
&
1
&
\begin{DUlineblock}{0em}
\item[] Byte order for interpreting data response.
\item[] \sphinxcode{\sphinxupquote{msb}} (\sphinxstyleemphasis{default}) most significant byte
\item[] \sphinxcode{\sphinxupquote{lsb}} least significant byte
\end{DUlineblock}
\\
\hline\begin{itemize}
\item {} 
wordOrder

\end{itemize}
&
enum
&
Y
&
Y
&
1
&
\begin{DUlineblock}{0em}
\item[] Word order (16-bit blocks) for interpreting data response.
\item[] \sphinxcode{\sphinxupquote{msw}} (\sphinxstyleemphasis{default}) most significant word
\item[] \sphinxcode{\sphinxupquote{lsw}} least significant word
\end{DUlineblock}
\\
\hline
\end{tabular}
\par
\sphinxattableend\end{savenotes}


\paragraph{writeData (MIN 22)}
\label{\detokenize{otaapi:writedata-min-22}}\label{\detokenize{otaapi:writedata}}
Writes select blocks of Modbus addresses directly.  (Must be \sphinxstyleemphasis{Output Coils} or \sphinxstyleemphasis{Holding Registers})


\begin{savenotes}\sphinxattablestart
\centering
\sphinxcapstartof{table}
\sphinxcaption{writeData}\label{\detokenize{otaapi:id16}}
\sphinxaftercaption
\begin{tabular}[t]{|\X{25}{100}|\X{10}{100}|\X{5}{100}|\X{5}{100}|\X{5}{100}|\X{50}{100}|}
\hline
\sphinxstyletheadfamily 
Field Name
&\sphinxstyletheadfamily 
Type
&\sphinxstyletheadfamily 
Optional
&\sphinxstyletheadfamily 
Fixed
&\sphinxstyletheadfamily 
Size
&\sphinxstyletheadfamily 
Comments
\\
\hline
addresses
&
array
&
N
&
N
&
100
&
The list of Modbus register addresses to be read.
\\
\hline\begin{itemize}
\item {} 
deviceId

\end{itemize}
&
unsignedInt
&
N
&
Y
&
8
&
Unique identifier of the \sphinxstylestrong{device} in the configuration.
\\
\hline\begin{itemize}
\item {} 
registerType

\end{itemize}
&
enum
&
N
&
Y
&
2
&
\begin{DUlineblock}{0em}
\item[] The register type:
\item[] \sphinxcode{\sphinxupquote{coil}}
\item[] \sphinxcode{\sphinxupquote{input}}
\item[] \sphinxcode{\sphinxupquote{inputRegister}}
\item[] \sphinxcode{\sphinxupquote{holdingRegister}}
\end{DUlineblock}
\\
\hline\begin{itemize}
\item {} 
address

\end{itemize}
&
unsignedInt
&
N
&
Y
&
31
&
The starting address to write.  (May be offset if \sphinxstylestrong{device} is configured with \sphinxstylestrong{plcBaseAddress})
\\
\hline\begin{itemize}
\item {} 
encoding

\end{itemize}
&
enum
&
N
&
Y
&
4
&
\begin{DUlineblock}{0em}
\item[] Data conversion to apply to the submitted (\sphinxstyleemphasis{string}) \sphinxstylestrong{value} when writing to the register.
\item[] \sphinxcode{\sphinxupquote{boolean}}
\item[] \sphinxcode{\sphinxupquote{int8}}
\item[] \sphinxcode{\sphinxupquote{uInt8}}
\item[] \sphinxcode{\sphinxupquote{int16}}
\item[] \sphinxcode{\sphinxupquote{uInt16}}
\item[] \sphinxcode{\sphinxupquote{int32}}
\item[] \sphinxcode{\sphinxupquote{float32}}
\item[] \sphinxcode{\sphinxupquote{string}}
\item[] \sphinxcode{\sphinxupquote{base64string}}
\item[] \sphinxcode{\sphinxupquote{raw}}
\end{DUlineblock}
\\
\hline\begin{itemize}
\item {} 
value

\end{itemize}
&
string
&
N
&
N
&
255
&
The value to write, presented as an ASCII string.
\begin{itemize}
\item {} 
For sending raw data, the string must be represented as space separated hex values e.g. “0A 1B 2C”

\item {} 
For sending Boolean data, the string must be represented as a number e.g. “1” for \sphinxstyleemphasis{true}

\end{itemize}
\\
\hline\begin{itemize}
\item {} 
byteOrder

\end{itemize}
&
enum
&
Y
&
Y
&
1
&
\begin{DUlineblock}{0em}
\item[] Byte order for interpreting data response.
\item[] \sphinxcode{\sphinxupquote{msb}} (\sphinxstyleemphasis{default}) most significant byte
\item[] \sphinxcode{\sphinxupquote{lsb}} least significant byte
\end{DUlineblock}
\\
\hline\begin{itemize}
\item {} 
wordOrder

\end{itemize}
&
enum
&
Y
&
Y
&
1
&
\begin{DUlineblock}{0em}
\item[] Word order (16-bit blocks) for interpreting data response.
\item[] \sphinxcode{\sphinxupquote{msw}} (\sphinxstyleemphasis{default}) most significant word
\item[] \sphinxcode{\sphinxupquote{lsw}} least significant word
\end{DUlineblock}
\\
\hline
\end{tabular}
\par
\sphinxattableend\end{savenotes}


\paragraph{setRefPosition (MIN 23)}
\label{\detokenize{otaapi:setrefposition-min-23}}\label{\detokenize{otaapi:setrefposition}}
Acknowledges the receipt of a {\hyperref[\detokenize{otaapi:positionchangealert}]{\sphinxcrossref{\DUrole{std,std-ref}{positionChangeAlert (MIN 104)}}}} and sets the current location as the new movement reference.

This message contains no \sphinxstylestrong{Fields}, it is just submitted with \sphinxstylestrong{SIN} and \sphinxstylestrong{MIN}.


\subsubsection{From-Mobile Messages}
\label{\detokenize{otaapi:from-mobile-messages}}\label{\detokenize{otaapi:id2}}

\begin{savenotes}\sphinxattablestart
\centering
\sphinxcapstartof{table}
\sphinxcaption{From-Mobile Messages}\label{\detokenize{otaapi:id17}}
\sphinxaftercaption
\begin{tabular}[t]{|\X{10}{105}|\X{25}{105}|\X{70}{105}|}
\hline
\sphinxstyletheadfamily 
MIN
&\sphinxstyletheadfamily 
Name
&\sphinxstyletheadfamily 
Description
\\
\hline
10
&
{\hyperref[\detokenize{otaapi:modbusconfig}]{\sphinxcrossref{\DUrole{std,std-ref}{modbusConfig}}}}
&
Response to set/get/del {\hyperref[\detokenize{otaapi:setmodbusconfig}]{\sphinxcrossref{\DUrole{std,std-ref}{ModbusConfig}}}}
\\
\hline
11
&
{\hyperref[\detokenize{otaapi:reportconfig}]{\sphinxcrossref{\DUrole{std,std-ref}{reportConfig}}}}
&
Response to set/get/del {\hyperref[\detokenize{otaapi:setreportconfig}]{\sphinxcrossref{\DUrole{std,std-ref}{ReportConfig}}}}
\\
\hline
12
&
{\hyperref[\detokenize{otaapi:alertconfig}]{\sphinxcrossref{\DUrole{std,std-ref}{alertConfig}}}}
&
Response to set/get/del {\hyperref[\detokenize{otaapi:setalertconfig}]{\sphinxcrossref{\DUrole{std,std-ref}{AlertConfig}}}}
\\
\hline
13
&
{\hyperref[\detokenize{otaapi:readparamresults}]{\sphinxcrossref{\DUrole{std,std-ref}{readParamResults}}}}
&
Response to {\hyperref[\detokenize{otaapi:readparam}]{\sphinxcrossref{\DUrole{std,std-ref}{readParam}}}}
\\
\hline
14
&
{\hyperref[\detokenize{otaapi:writeparamresults}]{\sphinxcrossref{\DUrole{std,std-ref}{writeParamResults}}}}
&
Response to {\hyperref[\detokenize{otaapi:writeparam}]{\sphinxcrossref{\DUrole{std,std-ref}{writeParam}}}}
\\
\hline
15
&
{\hyperref[\detokenize{otaapi:readdataresults}]{\sphinxcrossref{\DUrole{std,std-ref}{readDataResults}}}}
&
Response to {\hyperref[\detokenize{otaapi:readdata}]{\sphinxcrossref{\DUrole{std,std-ref}{readData}}}}
\\
\hline
16
&
{\hyperref[\detokenize{otaapi:writedataresults}]{\sphinxcrossref{\DUrole{std,std-ref}{writeDataResults}}}}
&
Response to {\hyperref[\detokenize{otaapi:writedata}]{\sphinxcrossref{\DUrole{std,std-ref}{writeData}}}}
\\
\hline
100
&
{\hyperref[\detokenize{otaapi:periodicreport}]{\sphinxcrossref{\DUrole{std,std-ref}{periodicReport}}}}
&
Periodic \sphinxstylestrong{report} generated at time interval or time of day.
\\
\hline
101
&
{\hyperref[\detokenize{otaapi:paramalerton}]{\sphinxcrossref{\DUrole{std,std-ref}{paramAlertON}}}}
&
Immediate \sphinxstylestrong{alert} generated by high/low threshold trigger.
\\
\hline
102
&
{\hyperref[\detokenize{otaapi:paramalertoff}]{\sphinxcrossref{\DUrole{std,std-ref}{paramAlertOFF}}}}
&
Immediate \sphinxstylestrong{alert} generated by return to normal from outside threshold.
\\
\hline
103
&
{\hyperref[\detokenize{otaapi:paramalertchange}]{\sphinxcrossref{\DUrole{std,std-ref}{paramAlertChange}}}}
&
Immediate \sphinxstylestrong{alert} generated by change threshold trigger.
\\
\hline
104
&
{\hyperref[\detokenize{otaapi:positionchangealert}]{\sphinxcrossref{\DUrole{std,std-ref}{positionChangeAlert}}}}
&
Immediate \sphinxstylestrong{alert} based on movement detection/geolocation.
\\
\hline
200
&
{\hyperref[\detokenize{otaapi:error}]{\sphinxcrossref{\DUrole{std,std-ref}{error}}}}
&
Error alert due to Modbus/serial or configuration problem.
\\
\hline
\end{tabular}
\par
\sphinxattableend\end{savenotes}


\bigskip\hrule\bigskip



\paragraph{modbusConfig (MIN 10)}
\label{\detokenize{otaapi:modbusconfig-min-10}}\label{\detokenize{otaapi:modbusconfig}}
Response to {\hyperref[\detokenize{otaapi:setmodbusconfig}]{\sphinxcrossref{\DUrole{std,std-ref}{setModbusConfig (MIN 10)}}}} or {\hyperref[\detokenize{otaapi:getmodbusconfig}]{\sphinxcrossref{\DUrole{std,std-ref}{getModbusConfig (MIN 11)}}}} or {\hyperref[\detokenize{otaapi:delmodbusconfig}]{\sphinxcrossref{\DUrole{std,std-ref}{delModbusConfig (MIN 12)}}}}.


\begin{savenotes}\sphinxattablestart
\centering
\sphinxcapstartof{table}
\sphinxcaption{modbusConfig}\label{\detokenize{otaapi:id18}}
\sphinxaftercaption
\begin{tabular}[t]{|\X{25}{100}|\X{10}{100}|\X{5}{100}|\X{5}{100}|\X{5}{100}|\X{50}{100}|}
\hline
\sphinxstyletheadfamily 
Field Name
&\sphinxstyletheadfamily 
Type
&\sphinxstyletheadfamily 
Optional
&\sphinxstyletheadfamily 
Fixed
&\sphinxstyletheadfamily 
Size
&\sphinxstyletheadfamily 
Comments
\\
\hline
ports
&
array
&
Y
&
Y
&
3
&
Included only if the set/get operation specified \sphinxstylestrong{ports}.
\\
\hline\begin{itemize}
\item {} 
port

\end{itemize}
&
enum
&
N
&
N/A
&
2
&
\begin{DUlineblock}{0em}
\item[] Name of the port
\item[] \sphinxcode{\sphinxupquote{rs232}}
\item[] \sphinxcode{\sphinxupquote{rs485}}
\item[] \sphinxcode{\sphinxupquote{rs232aux}}
\end{DUlineblock}
\\
\hline\begin{itemize}
\item {} 
baudRate

\end{itemize}
&
enum
&
N
&
N/A
&
3
&
\begin{DUlineblock}{0em}
\item[] Speed of serial communications.
\item[] \sphinxcode{\sphinxupquote{1200}}, \sphinxcode{\sphinxupquote{2400}}, \sphinxcode{\sphinxupquote{4800}}, \sphinxcode{\sphinxupquote{9600}}, \sphinxcode{\sphinxupquote{19200}}, \sphinxcode{\sphinxupquote{38400}}, \sphinxcode{\sphinxupquote{57600}}, \sphinxcode{\sphinxupquote{115200}}
\end{DUlineblock}
\\
\hline\begin{itemize}
\item {} 
parity

\end{itemize}
&
enum
&
N
&
N/A
&
2
&
\begin{DUlineblock}{0em}
\item[] \sphinxcode{\sphinxupquote{none}}
\item[] \sphinxcode{\sphinxupquote{even}}
\item[] \sphinxcode{\sphinxupquote{odd}}
\end{DUlineblock}
\\
\hline\begin{itemize}
\item {} 
mode

\end{itemize}
&
enum
&
N
&
N/A
&
1
&
\sphinxcode{\sphinxupquote{rtu}} or \sphinxcode{\sphinxupquote{ascii}}
\\
\hline
devices
&
array
&
Y
&
N
&
100
&
Included only if the set/get operation specified \sphinxstylestrong{devices}.
\\
\hline\begin{itemize}
\item {} 
deviceId

\end{itemize}
&
unsignedInt
&
N
&
N/A
&
8
&
Unique identifier of the device in the Modbus service.
\\
\hline\begin{itemize}
\item {} 
networkId

\end{itemize}
&
unsignedInt
&
N
&
N/A
&
8
&
Modbus Slave address on the physical network.
\\
\hline\begin{itemize}
\item {} 
port

\end{itemize}
&
enum
&
N
&
N/A
&
2
&
Serial port on which the device is attached.
\\
\hline\begin{itemize}
\item {} 
pollInterval

\end{itemize}
&
unsignedInt
&
N
&
N/A
&
17
&
The interval (seconds) at which the device registers are polled.
\\
\hline\begin{itemize}
\item {} 
byteOrder

\end{itemize}
&
enum
&
N
&
N/A
&
1
&
\sphinxcode{\sphinxupquote{msb}} or \sphinxcode{\sphinxupquote{lsb}}
\\
\hline\begin{itemize}
\item {} 
wordOrder

\end{itemize}
&
enum
&
N
&
N/A
&
1
&
\sphinxcode{\sphinxupquote{msw}} or \sphinxcode{\sphinxupquote{lsw}}
\\
\hline\begin{itemize}
\item {} 
serialRxTimeout

\end{itemize}
&
unsignedInt
&
N
&
N/A
&
15
&
Read timeout (milliseconds) waiting on next byte to be received.
\\
\hline\begin{itemize}
\item {} 
serialTxTimeout

\end{itemize}
&
unsignedInt
&
N
&
N/A
&
15
&
Read timeout (milliseconds) waiting on last transmitted byte to be confirmed.
\\
\hline\begin{itemize}
\item {} 
plcBaseAddress

\end{itemize}
&
enum
&
N
&
N/A
&
1
&
1 = Use PLC addressing (non zero-based)
\\
\hline\begin{itemize}
\item {} 
retries

\end{itemize}
&
unsignedInt
&
N
&
N/A
&
4
&
The number of serial retries to attempt before raising a communication error.
\\
\hline
parameters
&
array
&
Y
&
N
&
300
&
Included only if the set/get operation specified \sphinxstylestrong{parameters}.
\\
\hline\begin{itemize}
\item {} 
paramId

\end{itemize}
&
unsignedInt
&
N
&
N/A
&
10
&
Unique identifier of the parameter used for reporting and analytics.
\\
\hline\begin{itemize}
\item {} 
deviceId

\end{itemize}
&
unsignedInt
&
N
&
N/A
&
8
&
Unique identifier of the device in the Modbus service.
\\
\hline\begin{itemize}
\item {} 
registerType

\end{itemize}
&
enum
&
N
&
N/A
&
2
&
\begin{DUlineblock}{0em}
\item[] \sphinxcode{\sphinxupquote{coil}} - discrete output coil
\item[] \sphinxcode{\sphinxupquote{input}} - discrete digital input
\item[] \sphinxcode{\sphinxupquote{analog}} - (analog) input register
\item[] \sphinxcode{\sphinxupquote{holding}} - holding register
\end{DUlineblock}
\\
\hline\begin{itemize}
\item {} 
encoding

\end{itemize}
&
enum
&
N
&
N/A
&
4
&
\sphinxcode{\sphinxupquote{boolean}}, \sphinxcode{\sphinxupquote{int8}}, \sphinxcode{\sphinxupquote{uint8}}, \sphinxcode{\sphinxupquote{int16}}, \sphinxcode{\sphinxupquote{uint16}}, \sphinxcode{\sphinxupquote{int32}}, \sphinxcode{\sphinxupquote{float32}}, \sphinxcode{\sphinxupquote{string}}, \sphinxcode{\sphinxupquote{base64string}}, \sphinxcode{\sphinxupquote{raw}}
\\
\hline\begin{itemize}
\item {} 
address

\end{itemize}
&
unsignedInt
&
N
&
N/A
&
31
&
Starting Modbus register address of the data
\\
\hline\begin{itemize}
\item {} 
length

\end{itemize}
&
unsignedInt
&
N
&
N/A
&
12
&
Number of registers used to contain the data (including the starting address)
\\
\hline\begin{itemize}
\item {} 
mult

\end{itemize}
&
unsignedInt
&
N
&
N/A
&
7
&
A multiplier applied to the data value prior to reporting or analytics.
\\
\hline
properties
&
array
&
Y
&
N/A
&
245
&
Included only if the set/get operation specified \sphinxstylestrong{properties}.
\\
\hline\begin{itemize}
\item {} 
paramId

\end{itemize}
&
unsignedInt
&
N
&
N/A
&
10
&
Unique identifier of the parameter used for reporting and analytics.
\\
\hline\begin{itemize}
\item {} 
pin

\end{itemize}
&
unsignedInt
&
N
&
N/A
&
8
&
The property identification number (within Modbus service) to hold the parameter value.
\\
\hline
\end{tabular}
\par
\sphinxattableend\end{savenotes}


\paragraph{reportConfig (MIN 11)}
\label{\detokenize{otaapi:reportconfig-min-11}}\label{\detokenize{otaapi:reportconfig}}
Response to {\hyperref[\detokenize{otaapi:setreportconfig}]{\sphinxcrossref{\DUrole{std,std-ref}{setReportConfig (MIN 13)}}}} or {\hyperref[\detokenize{otaapi:getreportconfig}]{\sphinxcrossref{\DUrole{std,std-ref}{getReportConfig (MIN 14)}}}} or {\hyperref[\detokenize{otaapi:delreportconfig}]{\sphinxcrossref{\DUrole{std,std-ref}{delReportConfig (MIN 15)}}}}.


\begin{savenotes}\sphinxattablestart
\centering
\sphinxcapstartof{table}
\sphinxcaption{reportConfig}\label{\detokenize{otaapi:id19}}
\sphinxaftercaption
\begin{tabular}[t]{|\X{25}{100}|\X{10}{100}|\X{5}{100}|\X{5}{100}|\X{5}{100}|\X{50}{100}|}
\hline
\sphinxstyletheadfamily 
Field Name
&\sphinxstyletheadfamily 
Type
&\sphinxstyletheadfamily 
Optional
&\sphinxstyletheadfamily 
Fixed
&\sphinxstyletheadfamily 
Size
&\sphinxstyletheadfamily 
Comments
\\
\hline
reports
&
array
&
N
&
N
&
100
&
The list of periodic \sphinxstylestrong{reports} configured (\sphinxstyleemphasis{set}) or requested (\sphinxstyleemphasis{get}).
\\
\hline\begin{itemize}
\item {} 
reportId

\end{itemize}
&
unsignedInt
&
N
&
Y
&
31
&
Unique identifier of the \sphinxstylestrong{report} associated with particular parameter values.
\\
\hline\begin{itemize}
\item {} 
paramIds

\end{itemize}
&
string
&
N
&
N
&
255
&
Comma-separated list of \sphinxstylestrong{parameter} ids to be included in the \sphinxstylestrong{report}.
\\
\hline\begin{itemize}
\item {} 
interval

\end{itemize}
&
unsignedInt
&
N
&
Y
&
20
&
The periodic \sphinxstylestrong{report} interval, in \sphinxstyleemphasis{seconds}.
\\
\hline\begin{itemize}
\item {} 
timeOfDay

\end{itemize}
&
unsignedInt
&
Y
&
Y
&
18
&
\sphinxstyleemphasis{Optional} time of day, in seconds (0..86400) to create a snapshot that will be reported within the \sphinxstylestrong{timeOfDayWindow}.
\\
\hline\begin{itemize}
\item {} 
timeOfDayWindow

\end{itemize}
&
unsignedInt
&
Y
&
Y
&
13
&
Included if using \sphinxstylestrong{timeOfDay}.  Window, in seconds, within which the \sphinxstylestrong{report} will be sent offset relative to the \sphinxstylestrong{timeOfDay} snapshot.
\\
\hline\begin{itemize}
\item {} 
msgBitMap

\end{itemize}
&
string
&
Y
&
Y
&
2
&
ASCII string of \sphinxstyleemphasis{metadata} fields to include, overriding \sphinxstylestrong{defaultMsgBitmap} value.  See {\hyperref[\detokenize{configuration:msgbitmap}]{\sphinxcrossref{\DUrole{std,std-ref}{Optional Metadata Fields Configuration (MsgBitmap, defaultMsgBitmap)}}}}.
\\
\hline
\end{tabular}
\par
\sphinxattableend\end{savenotes}


\paragraph{alertConfig (MIN 12)}
\label{\detokenize{otaapi:alertconfig-min-12}}\label{\detokenize{otaapi:alertconfig}}
Response to {\hyperref[\detokenize{otaapi:setalertconfig}]{\sphinxcrossref{\DUrole{std,std-ref}{setAlertConfig (MIN 16)}}}} or {\hyperref[\detokenize{otaapi:getalertconfig}]{\sphinxcrossref{\DUrole{std,std-ref}{getAlertConfig (MIN 17)}}}} or {\hyperref[\detokenize{otaapi:delalertconfig}]{\sphinxcrossref{\DUrole{std,std-ref}{delAlertConfig (MIN 18)}}}}.


\begin{savenotes}\sphinxattablestart
\centering
\sphinxcapstartof{table}
\sphinxcaption{alertConfig}\label{\detokenize{otaapi:id20}}
\sphinxaftercaption
\begin{tabular}[t]{|\X{25}{100}|\X{10}{100}|\X{5}{100}|\X{5}{100}|\X{5}{100}|\X{50}{100}|}
\hline
\sphinxstyletheadfamily 
Field Name
&\sphinxstyletheadfamily 
Type
&\sphinxstyletheadfamily 
Optional
&\sphinxstyletheadfamily 
Fixed
&\sphinxstyletheadfamily 
Size
&\sphinxstyletheadfamily 
Comments
\\
\hline
alerts
&
array
&
N
&
N
&
100
&
The list of \sphinxstylestrong{alerts} configured (\sphinxstyleemphasis{set}) or requested (\sphinxstyleemphasis{get}).
\\
\hline\begin{itemize}
\item {} 
alertId

\end{itemize}
&
unsignedInt
&
N
&
Y
&
31
&
Unique identifier of the \sphinxstylestrong{alert} associated with a particular parameter trigger.
\\
\hline\begin{itemize}
\item {} 
paramId

\end{itemize}
&
unsignedInt
&
N
&
Y
&
10
&
The ID of the trigger \sphinxstylestrong{parameter} on which the analytics is run.
\\
\hline\begin{itemize}
\item {} 
paramIds

\end{itemize}
&
string
&
Y
&
N
&
255
&
Comma-separated list of \sphinxstylestrong{parameter} values to be included in the \sphinxstylestrong{alert}, in addition to the trigger parameter.
\\
\hline\begin{itemize}
\item {} 
minON

\end{itemize}
&
int
&
Y
&
Y
&
32
&
The lower threshold against which the parameter is tested.  If the value drops below this, an \sphinxstylestrong{alertON} is sent.  The parameter stays in the alert state until the value rises above \sphinxstylestrong{minOFF}.
\\
\hline\begin{itemize}
\item {} 
minOFF

\end{itemize}
&
int
&
Y
&
Y
&
32
&
The normal low threshold against which the parameter is tested.  If the parameter is in the alert state and rises above this value, the alert state is off and an \sphinxstylestrong{alertOFF} is sent.

If \sphinxstyleemphasis{not} specified, \sphinxstylestrong{minON} will be used as the return to normal trigger.
\\
\hline\begin{itemize}
\item {} 
maxON

\end{itemize}
&
int
&
Y
&
Y
&
32
&
The upper threshold against which the parameter is tested.  If the value rises above this, an \sphinxstylestrong{alertON} is sent.  The parameter stays in the alert state until the value drops below \sphinxstylestrong{maxOFF}.
\\
\hline\begin{itemize}
\item {} 
maxOFF

\end{itemize}
&
int
&
Y
&
Y
&
32
&
The normal high threshold against which the parameter is tested.  If the parameter is in the alert state and drops below this value, the alert state is off and an \sphinxstylestrong{alertOFF} is sent.

If \sphinxstyleemphasis{not} specified, \sphinxstylestrong{maxON} will be used as the return to normal trigger.
\\
\hline\begin{itemize}
\item {} 
change

\end{itemize}
&
unsignedInt
&
Y
&
Y
&
31
&
The absolute value difference from the last reported value, which will trigger a \sphinxstylestrong{alertChange} to be sent.
\\
\hline\begin{itemize}
\item {} 
msgBitMap

\end{itemize}
&
string
&
Y
&
Y
&
2
&
ASCII string of \sphinxstyleemphasis{metadata} fields to include, overriding \sphinxstylestrong{defaultMsgBitmap} value.  See {\hyperref[\detokenize{configuration:msgbitmap}]{\sphinxcrossref{\DUrole{std,std-ref}{Optional Metadata Fields Configuration (MsgBitmap, defaultMsgBitmap)}}}}.
\\
\hline
\end{tabular}
\par
\sphinxattableend\end{savenotes}


\paragraph{readParamResults (MIN 13)}
\label{\detokenize{otaapi:readparamresults-min-13}}\label{\detokenize{otaapi:readparamresults}}
Response to {\hyperref[\detokenize{otaapi:readparam}]{\sphinxcrossref{\DUrole{std,std-ref}{readParam (MIN 19)}}}}.


\begin{savenotes}\sphinxattablestart
\centering
\sphinxcapstartof{table}
\sphinxcaption{readParamResults}\label{\detokenize{otaapi:id21}}
\sphinxaftercaption
\begin{tabular}[t]{|\X{25}{100}|\X{10}{100}|\X{5}{100}|\X{5}{100}|\X{5}{100}|\X{50}{100}|}
\hline
\sphinxstyletheadfamily 
Field Name
&\sphinxstyletheadfamily 
Type
&\sphinxstyletheadfamily 
Optional
&\sphinxstyletheadfamily 
Fixed
&\sphinxstyletheadfamily 
Size
&\sphinxstyletheadfamily 
Comments
\\
\hline
parameters
&
array
&
N
&
N
&
100
&
The list of \sphinxstylestrong{parameter} values being queried.
\\
\hline\begin{itemize}
\item {} 
paramId

\end{itemize}
&
unsignedInt
&
N
&
Y
&
10
&
Unique identifier of the \sphinxstylestrong{parameter} to query.
\\
\hline\begin{itemize}
\item {} 
value

\end{itemize}
&
dynamic
\begin{itemize}
\item {} 
unsignedInt

\item {} 
signedInt

\item {} 
string

\item {} 
data

\end{itemize}
&
N
&
N/A
&
N/A
&
Value with data type corresponding to the parameter encoding:
\begin{itemize}
\item {} 
\sphinxcode{\sphinxupquote{uint8}}, \sphinxcode{\sphinxupquote{uint16}} use \sphinxstyleemphasis{unsignedInt}

\item {} 
\sphinxcode{\sphinxupquote{int8}}, \sphinxcode{\sphinxupquote{int16}}, \sphinxcode{\sphinxupquote{int32}} use \sphinxstyleemphasis{signedInt}

\item {} 
\sphinxcode{\sphinxupquote{float32}}, \sphinxcode{\sphinxupquote{string}}, \sphinxcode{\sphinxupquote{base64string}} use \sphinxstyleemphasis{string}

\item {} 
\sphinxcode{\sphinxupquote{raw}} uses \sphinxstyleemphasis{data}

\end{itemize}
\\
\hline\begin{itemize}
\item {} 
paramTimestamp

\end{itemize}
&
unsignedInt
&
Y
&
N/A
&
31
&
Parameter timestamp in seconds since \sphinxstyleemphasis{1970-01-01T00:00:00Z}.

Only present if \sphinxcode{\sphinxupquote{paramTimestamp}} is configured in \sphinxstyleemphasis{metadata} options.
\\
\hline\begin{itemize}
\item {} 
errText

\end{itemize}
&
string
&
Y
&
N
&
100
&
Verbose error text if a read error occurred.
\\
\hline
timestamp
&
unsignedInt
&
Y
&
Y
&
31
&
Response timestamp, seconds since \sphinxstyleemphasis{1970-01-01T00:00:00Z}.

Only present if configured in \sphinxstyleemphasis{metadata} options.
\\
\hline
\end{tabular}
\par
\sphinxattableend\end{savenotes}


\paragraph{writeParamResults (MIN 14)}
\label{\detokenize{otaapi:writeparamresults-min-14}}\label{\detokenize{otaapi:writeparamresults}}
Response to {\hyperref[\detokenize{otaapi:writeparam}]{\sphinxcrossref{\DUrole{std,std-ref}{writeParam (MIN 20)}}}}.


\begin{savenotes}\sphinxattablestart
\centering
\sphinxcapstartof{table}
\sphinxcaption{writeParamResults}\label{\detokenize{otaapi:id22}}
\sphinxaftercaption
\begin{tabular}[t]{|\X{25}{100}|\X{10}{100}|\X{5}{100}|\X{5}{100}|\X{5}{100}|\X{50}{100}|}
\hline
\sphinxstyletheadfamily 
Field Name
&\sphinxstyletheadfamily 
Type
&\sphinxstyletheadfamily 
Optional
&\sphinxstyletheadfamily 
Fixed
&\sphinxstyletheadfamily 
Size
&\sphinxstyletheadfamily 
Comments
\\
\hline
parameters
&
array
&
N
&
N
&
100
&
The list of \sphinxstylestrong{parameter} values being queried.
\\
\hline\begin{itemize}
\item {} 
paramId

\end{itemize}
&
unsignedInt
&
N
&
Y
&
10
&
Unique identifier of the \sphinxstylestrong{parameter} to query.
\\
\hline\begin{itemize}
\item {} 
paramTimestamp

\end{itemize}
&
unsignedInt
&
Y
&
N/A
&
31
&
Parameter timestamp in seconds since \sphinxstyleemphasis{1970-01-01T00:00:00Z}.

Only present if \sphinxcode{\sphinxupquote{paramTimestamp}} is configured in \sphinxstyleemphasis{metadata} options.
\\
\hline\begin{itemize}
\item {} 
errText

\end{itemize}
&
string
&
Y
&
N
&
100
&
Verbose error text if a read error occurred.
\\
\hline
timestamp
&
unsignedInt
&
Y
&
Y
&
31
&
Response timestamp, seconds since \sphinxstyleemphasis{1970-01-01T00:00:00Z}.

Only present if configured in \sphinxstyleemphasis{metadata} options.
\\
\hline
\end{tabular}
\par
\sphinxattableend\end{savenotes}


\paragraph{readDataResults (MIN 15)}
\label{\detokenize{otaapi:readdataresults-min-15}}\label{\detokenize{otaapi:readdataresults}}
Response to {\hyperref[\detokenize{otaapi:readdata}]{\sphinxcrossref{\DUrole{std,std-ref}{readData (MIN 21)}}}}.


\begin{savenotes}\sphinxattablestart
\centering
\sphinxcapstartof{table}
\sphinxcaption{readDataResults}\label{\detokenize{otaapi:id23}}
\sphinxaftercaption
\begin{tabular}[t]{|\X{25}{100}|\X{10}{100}|\X{5}{100}|\X{5}{100}|\X{5}{100}|\X{50}{100}|}
\hline
\sphinxstyletheadfamily 
Field Name
&\sphinxstyletheadfamily 
Type
&\sphinxstyletheadfamily 
Optional
&\sphinxstyletheadfamily 
Fixed
&\sphinxstyletheadfamily 
Size
&\sphinxstyletheadfamily 
Comments
\\
\hline
results
&
array
&
N
&
N
&
100
&
The list of values being queried.
\\
\hline\begin{itemize}
\item {} 
deviceId

\end{itemize}
&
unsignedInt
&
N
&
Y
&
8
&
Unique identifier of the \sphinxstylestrong{device} in the configuration.
\\
\hline\begin{itemize}
\item {} 
registerType

\end{itemize}
&
enum
&
N
&
Y
&
2
&
\begin{DUlineblock}{0em}
\item[] The register type:
\item[] \sphinxcode{\sphinxupquote{coil}}
\item[] \sphinxcode{\sphinxupquote{input}}
\item[] \sphinxcode{\sphinxupquote{inputRegister}}
\item[] \sphinxcode{\sphinxupquote{holdingRegister}}
\end{DUlineblock}
\\
\hline\begin{itemize}
\item {} 
address

\end{itemize}
&
unsignedInt
&
N
&
Y
&
31
&
The starting address to read.  (May be offset if \sphinxstylestrong{device} is configured with \sphinxstylestrong{plcBaseAddress})
\\
\hline\begin{itemize}
\item {} 
encoding

\end{itemize}
&
enum
&
N
&
Y
&
4
&
\begin{DUlineblock}{0em}
\item[] Data conversion to apply to the requested value when responding.
\item[] \sphinxcode{\sphinxupquote{boolean}}
\item[] \sphinxcode{\sphinxupquote{int8}}
\item[] \sphinxcode{\sphinxupquote{uInt8}}
\item[] \sphinxcode{\sphinxupquote{int16}}
\item[] \sphinxcode{\sphinxupquote{uInt16}}
\item[] \sphinxcode{\sphinxupquote{int32}}
\item[] \sphinxcode{\sphinxupquote{float32}}
\item[] \sphinxcode{\sphinxupquote{string}}
\item[] \sphinxcode{\sphinxupquote{base64string}}
\item[] \sphinxcode{\sphinxupquote{raw}}
\end{DUlineblock}
\\
\hline\begin{itemize}
\item {} 
value

\end{itemize}
&
dynamic
\begin{itemize}
\item {} 
unsignedInt

\item {} 
signedInt

\item {} 
string

\item {} 
data

\end{itemize}
&
N
&
N/A
&
N/A
&
Value with data type corresponding to the parameter encoding:
\begin{itemize}
\item {} 
\sphinxcode{\sphinxupquote{uint8}}, \sphinxcode{\sphinxupquote{uint16}} use \sphinxstyleemphasis{unsignedInt}

\item {} 
\sphinxcode{\sphinxupquote{int8}}, \sphinxcode{\sphinxupquote{int16}}, \sphinxcode{\sphinxupquote{int32}} use \sphinxstyleemphasis{signedInt}

\item {} 
\sphinxcode{\sphinxupquote{float32}}, \sphinxcode{\sphinxupquote{string}}, \sphinxcode{\sphinxupquote{base64string}} use \sphinxstyleemphasis{string}

\item {} 
\sphinxcode{\sphinxupquote{raw}} uses \sphinxstyleemphasis{data}

\end{itemize}
\\
\hline
timestamp
&
unsignedInt
&
Y
&
Y
&
31
&
Response timestamp, seconds since \sphinxstyleemphasis{1970-01-01T00:00:00Z}.

Only present if configured in \sphinxstyleemphasis{metadata} options.
\\
\hline
\end{tabular}
\par
\sphinxattableend\end{savenotes}


\paragraph{writeDataResults (MIN 16)}
\label{\detokenize{otaapi:writedataresults-min-16}}\label{\detokenize{otaapi:writedataresults}}
Response to {\hyperref[\detokenize{otaapi:writedata}]{\sphinxcrossref{\DUrole{std,std-ref}{writeData (MIN 22)}}}}.


\begin{savenotes}\sphinxattablestart
\centering
\sphinxcapstartof{table}
\sphinxcaption{writeDataResults}\label{\detokenize{otaapi:id24}}
\sphinxaftercaption
\begin{tabular}[t]{|\X{25}{100}|\X{10}{100}|\X{5}{100}|\X{5}{100}|\X{5}{100}|\X{50}{100}|}
\hline
\sphinxstyletheadfamily 
Field Name
&\sphinxstyletheadfamily 
Type
&\sphinxstyletheadfamily 
Optional
&\sphinxstyletheadfamily 
Fixed
&\sphinxstyletheadfamily 
Size
&\sphinxstyletheadfamily 
Comments
\\
\hline
results
&
array
&
N
&
N
&
100
&
The list of values that were to be written.
\\
\hline\begin{itemize}
\item {} 
deviceId

\end{itemize}
&
unsignedInt
&
N
&
Y
&
8
&
Unique identifier of the \sphinxstylestrong{device} in the configuration.
\\
\hline\begin{itemize}
\item {} 
registerType

\end{itemize}
&
enum
&
N
&
Y
&
2
&
\begin{DUlineblock}{0em}
\item[] The register type:
\item[] \sphinxcode{\sphinxupquote{coil}}
\item[] \sphinxcode{\sphinxupquote{input}}
\item[] \sphinxcode{\sphinxupquote{inputRegister}}
\item[] \sphinxcode{\sphinxupquote{holdingRegister}}
\end{DUlineblock}
\\
\hline\begin{itemize}
\item {} 
address

\end{itemize}
&
unsignedInt
&
N
&
Y
&
31
&
The starting address to write.  (May be offset if \sphinxstylestrong{device} is configured with \sphinxstylestrong{plcBaseAddress})
\\
\hline\begin{itemize}
\item {} 
encoding

\end{itemize}
&
enum
&
N
&
Y
&
4
&
\begin{DUlineblock}{0em}
\item[] Data conversion to apply to the requested value when responding.
\item[] \sphinxcode{\sphinxupquote{boolean}}
\item[] \sphinxcode{\sphinxupquote{int8}}
\item[] \sphinxcode{\sphinxupquote{uInt8}}
\item[] \sphinxcode{\sphinxupquote{int16}}
\item[] \sphinxcode{\sphinxupquote{uInt16}}
\item[] \sphinxcode{\sphinxupquote{int32}}
\item[] \sphinxcode{\sphinxupquote{float32}}
\item[] \sphinxcode{\sphinxupquote{string}}
\item[] \sphinxcode{\sphinxupquote{base64string}}
\item[] \sphinxcode{\sphinxupquote{raw}}
\end{DUlineblock}
\\
\hline\begin{itemize}
\item {} 
writeTimestamp

\end{itemize}
&
unsignedInt
&
Y
&
N/A
&
31
&
Parameter timestamp in seconds since \sphinxstyleemphasis{1970-01-01T00:00:00Z}.

Only present if \sphinxcode{\sphinxupquote{paramTimestamp}} is configured in \sphinxstyleemphasis{metadata} options.
\\
\hline\begin{itemize}
\item {} 
errText

\end{itemize}
&
string
&
Y
&
N
&
100
&
Verbose error text if a read error occurred.
\\
\hline
timestamp
&
unsignedInt
&
Y
&
Y
&
31
&
Response timestamp, seconds since \sphinxstyleemphasis{1970-01-01T00:00:00Z}.

Only present if configured in \sphinxstyleemphasis{metadata} options.
\\
\hline
\end{tabular}
\par
\sphinxattableend\end{savenotes}


\paragraph{periodicReport (MIN 100)}
\label{\detokenize{otaapi:periodicreport-min-100}}\label{\detokenize{otaapi:periodicreport}}
Sent automatically at the prescribed time interval or offset from Time of Day.


\begin{savenotes}\sphinxattablestart
\centering
\sphinxcapstartof{table}
\sphinxcaption{periodicReport}\label{\detokenize{otaapi:id25}}
\sphinxaftercaption
\begin{tabular}[t]{|\X{25}{100}|\X{10}{100}|\X{5}{100}|\X{5}{100}|\X{5}{100}|\X{50}{100}|}
\hline
\sphinxstyletheadfamily 
Field Name
&\sphinxstyletheadfamily 
Type
&\sphinxstyletheadfamily 
Optional
&\sphinxstyletheadfamily 
Fixed
&\sphinxstyletheadfamily 
Size
&\sphinxstyletheadfamily 
Comments
\\
\hline
reportId
&
unsignedInt
&
N
&
Y
&
8
&
Unique identifer of the \sphinxstylestrong{report}.
\\
\hline
timestamp
&
unsignedInt
&
Y
&
Y
&
31
&
Report timestamp, seconds since \sphinxstyleemphasis{1970-01-01T00:00:00Z}.

Only present if configured in \sphinxstyleemphasis{metadata} options.
\\
\hline
parameters
&
array
&
N
&
N
&
100
&
A set of parameter values defined by the \sphinxstylestrong{report} configuration.
\\
\hline\begin{itemize}
\item {} 
paramId

\end{itemize}
&
unsignedInt
&
N
&
N/A
&
8
&
The \sphinxcode{\sphinxupquote{paramId}} being reported.
\\
\hline\begin{itemize}
\item {} 
value

\end{itemize}
&
dynamic
\begin{itemize}
\item {} 
unsignedInt

\item {} 
signedInt

\item {} 
string

\item {} 
data

\end{itemize}
&
N
&
N/A
&
N/A
&
Value with data type corresponding to the parameter encoding:
\begin{itemize}
\item {} 
\sphinxcode{\sphinxupquote{uint8}}, \sphinxcode{\sphinxupquote{uint16}} use \sphinxstyleemphasis{unsignedInt}

\item {} 
\sphinxcode{\sphinxupquote{int8}}, \sphinxcode{\sphinxupquote{int16}}, \sphinxcode{\sphinxupquote{int32}} use \sphinxstyleemphasis{signedInt}

\item {} 
\sphinxcode{\sphinxupquote{float32}}, \sphinxcode{\sphinxupquote{string}}, \sphinxcode{\sphinxupquote{base64string}} use \sphinxstyleemphasis{string}

\item {} 
\sphinxcode{\sphinxupquote{raw}} uses \sphinxstyleemphasis{data}

\end{itemize}
\\
\hline\begin{itemize}
\item {} 
paramTimestamp

\end{itemize}
&
unsignedInt
&
Y
&
N/A
&
31
&
Parameter timestamp in seconds since \sphinxstyleemphasis{1970-01-01T00:00:00Z}.

Only present if \sphinxcode{\sphinxupquote{paramTimestamp}} is configured in \sphinxstyleemphasis{metadata} options.
\\
\hline
latitude
&
signedInt
&
Y
&
Y
&
24
&
Latitude in milliminutes, if present in \sphinxstyleemphasis{metadata} options.

Divide by 60000 to get decimal degrees (round to 6 decimal places precision).
\\
\hline
longitude
&
signedInt
&
Y
&
Y
&
25
&
Longitude in milliminutes, if present in \sphinxstyleemphasis{metadata} options.

Divide by 60000 to get decimal degrees (round to 6 decimal places precision).
\\
\hline
altitude
&
signedInt
&
Y
&
Y
&
20
&
Altitude in meters above sea level, if present in \sphinxstyleemphasis{metadata} options.
\\
\hline
\end{tabular}
\par
\sphinxattableend\end{savenotes}


\paragraph{paramAlertON (MIN 101)}
\label{\detokenize{otaapi:paramalerton-min-101}}\label{\detokenize{otaapi:paramalerton}}
Sent automatically when an \sphinxstylestrong{alert} condition has been triggered and is outside normal operating range.


\begin{savenotes}\sphinxattablestart
\centering
\sphinxcapstartof{table}
\sphinxcaption{paramAlertON}\label{\detokenize{otaapi:id26}}
\sphinxaftercaption
\begin{tabular}[t]{|\X{25}{100}|\X{10}{100}|\X{5}{100}|\X{5}{100}|\X{5}{100}|\X{50}{100}|}
\hline
\sphinxstyletheadfamily 
Field Name
&\sphinxstyletheadfamily 
Type
&\sphinxstyletheadfamily 
Optional
&\sphinxstyletheadfamily 
Fixed
&\sphinxstyletheadfamily 
Size
&\sphinxstyletheadfamily 
Comments
\\
\hline
paramId
&
unsignedInt
&
N
&
Y
&
8
&
\sphinxstylestrong{Parameter} for which the \sphinxstylestrong{alert} has been raised.
\\
\hline
value
&
dynamic
\begin{itemize}
\item {} 
unsignedInt

\item {} 
signedInt

\item {} 
string

\item {} 
data

\end{itemize}
&
N
&
N/A
&
N/A
&
Value of the triggering parameter with data type corresponding to the parameter encoding:
\begin{itemize}
\item {} 
\sphinxcode{\sphinxupquote{uint8}}, \sphinxcode{\sphinxupquote{uint16}} use \sphinxstyleemphasis{unsignedInt}

\item {} 
\sphinxcode{\sphinxupquote{int8}}, \sphinxcode{\sphinxupquote{int16}}, \sphinxcode{\sphinxupquote{int32}} use \sphinxstyleemphasis{signedInt}

\item {} 
\sphinxcode{\sphinxupquote{float32}}, \sphinxcode{\sphinxupquote{string}}, \sphinxcode{\sphinxupquote{base64string}} use \sphinxstyleemphasis{string}

\item {} 
\sphinxcode{\sphinxupquote{raw}} uses \sphinxstyleemphasis{data}

\end{itemize}
\\
\hline
timestamp
&
unsignedInt
&
Y
&
Y
&
31
&
Alert timestamp, seconds since \sphinxstyleemphasis{1970-01-01T00:00:00Z}.

Only present if configured in \sphinxstyleemphasis{metadata} options.
\\
\hline
parameters
&
array
&
Y
&
N
&
100
&
A set of parameter values defined by the \sphinxstylestrong{alert} configuration.
\\
\hline\begin{itemize}
\item {} 
paramId

\end{itemize}
&
unsignedInt
&
N
&
N/A
&
8
&
The \sphinxcode{\sphinxupquote{paramId}} being reported.
\\
\hline\begin{itemize}
\item {} 
value

\end{itemize}
&
dynamic
\begin{itemize}
\item {} 
unsignedInt

\item {} 
signedInt

\item {} 
string

\item {} 
data

\end{itemize}
&
N
&
N/A
&
N/A
&
Value with data type corresponding to the parameter encoding:
\begin{itemize}
\item {} 
\sphinxcode{\sphinxupquote{uint8}}, \sphinxcode{\sphinxupquote{uint16}} use \sphinxstyleemphasis{unsignedInt}

\item {} 
\sphinxcode{\sphinxupquote{int8}}, \sphinxcode{\sphinxupquote{int16}}, \sphinxcode{\sphinxupquote{int32}} use \sphinxstyleemphasis{signedInt}

\item {} 
\sphinxcode{\sphinxupquote{float32}}, \sphinxcode{\sphinxupquote{string}}, \sphinxcode{\sphinxupquote{base64string}} use \sphinxstyleemphasis{string}

\item {} 
\sphinxcode{\sphinxupquote{raw}} uses \sphinxstyleemphasis{data}

\end{itemize}
\\
\hline\begin{itemize}
\item {} 
paramTimestamp

\end{itemize}
&
unsignedInt
&
Y
&
N/A
&
31
&
Parameter timestamp in seconds since \sphinxstyleemphasis{1970-01-01T00:00:00Z}.

Only present if \sphinxcode{\sphinxupquote{paramTimestamp}} is configured in \sphinxstyleemphasis{metadata} options.
\\
\hline
paramTimestamp
&
unsignedInt
&
Y
&
Y
&
31
&
Alert parameter timestamp, seconds since \sphinxstyleemphasis{1970-01-01T00:00:00Z}.

Only present if \sphinxcode{\sphinxupquote{paramTimestamp}} configured in \sphinxstyleemphasis{metadata} options.
\\
\hline
latitude
&
signedInt
&
Y
&
Y
&
24
&
Latitude in milliminutes, if present in \sphinxstyleemphasis{metadata} options.

Divide by 60000 to get decimal degrees (round to 6 decimal places precision).
\\
\hline
longitude
&
signedInt
&
Y
&
Y
&
25
&
Longitude in milliminutes, if present in \sphinxstyleemphasis{metadata} options.

Divide by 60000 to get decimal degrees (round to 6 decimal places precision).
\\
\hline
altitude
&
signedInt
&
Y
&
Y
&
20
&
Altitude in meters above sea level, if present in \sphinxstyleemphasis{metadata} options.
\\
\hline
\end{tabular}
\par
\sphinxattableend\end{savenotes}


\paragraph{paramAlertOFF (MIN 102)}
\label{\detokenize{otaapi:paramalertoff-min-102}}\label{\detokenize{otaapi:paramalertoff}}
Sent automatically when an \sphinxstylestrong{alert} condition has returned to the normal operating range.


\begin{savenotes}\sphinxattablestart
\centering
\sphinxcapstartof{table}
\sphinxcaption{paramAlertOFF}\label{\detokenize{otaapi:id27}}
\sphinxaftercaption
\begin{tabular}[t]{|\X{25}{100}|\X{10}{100}|\X{5}{100}|\X{5}{100}|\X{5}{100}|\X{50}{100}|}
\hline
\sphinxstyletheadfamily 
Field Name
&\sphinxstyletheadfamily 
Type
&\sphinxstyletheadfamily 
Optional
&\sphinxstyletheadfamily 
Fixed
&\sphinxstyletheadfamily 
Size
&\sphinxstyletheadfamily 
Comments
\\
\hline
paramId
&
unsignedInt
&
N
&
Y
&
8
&
\sphinxstylestrong{Parameter} for which the \sphinxstylestrong{alert} has been raised.
\\
\hline
value
&
dynamic
\begin{itemize}
\item {} 
unsignedInt

\item {} 
signedInt

\item {} 
string

\item {} 
data

\end{itemize}
&
N
&
N/A
&
N/A
&
Value of the triggering parameter with data type corresponding to the parameter encoding:
\begin{itemize}
\item {} 
\sphinxcode{\sphinxupquote{uint8}}, \sphinxcode{\sphinxupquote{uint16}} use \sphinxstyleemphasis{unsignedInt}

\item {} 
\sphinxcode{\sphinxupquote{int8}}, \sphinxcode{\sphinxupquote{int16}}, \sphinxcode{\sphinxupquote{int32}} use \sphinxstyleemphasis{signedInt}

\item {} 
\sphinxcode{\sphinxupquote{float32}}, \sphinxcode{\sphinxupquote{string}}, \sphinxcode{\sphinxupquote{base64string}} use \sphinxstyleemphasis{string}

\item {} 
\sphinxcode{\sphinxupquote{raw}} uses \sphinxstyleemphasis{data}

\end{itemize}
\\
\hline
timestamp
&
unsignedInt
&
Y
&
Y
&
31
&
Alert timestamp, seconds since \sphinxstyleemphasis{1970-01-01T00:00:00Z}.

Only present if configured in \sphinxstyleemphasis{metadata} options.
\\
\hline
parameters
&
array
&
Y
&
N
&
100
&
A set of parameter values defined by the \sphinxstylestrong{alert} configuration.
\\
\hline\begin{itemize}
\item {} 
paramId

\end{itemize}
&
unsignedInt
&
N
&
N/A
&
8
&
The \sphinxcode{\sphinxupquote{paramId}} being reported.
\\
\hline\begin{itemize}
\item {} 
value

\end{itemize}
&
dynamic
\begin{itemize}
\item {} 
unsignedInt

\item {} 
signedInt

\item {} 
string

\item {} 
data

\end{itemize}
&
N
&
N/A
&
N/A
&
Value with data type corresponding to the parameter encoding:
\begin{itemize}
\item {} 
\sphinxcode{\sphinxupquote{uint8}}, \sphinxcode{\sphinxupquote{uint16}} use \sphinxstyleemphasis{unsignedInt}

\item {} 
\sphinxcode{\sphinxupquote{int8}}, \sphinxcode{\sphinxupquote{int16}}, \sphinxcode{\sphinxupquote{int32}} use \sphinxstyleemphasis{signedInt}

\item {} 
\sphinxcode{\sphinxupquote{float32}}, \sphinxcode{\sphinxupquote{string}}, \sphinxcode{\sphinxupquote{base64string}} use \sphinxstyleemphasis{string}

\item {} 
\sphinxcode{\sphinxupquote{raw}} uses \sphinxstyleemphasis{data}

\end{itemize}
\\
\hline\begin{itemize}
\item {} 
paramTimestamp

\end{itemize}
&
unsignedInt
&
Y
&
N/A
&
31
&
Parameter timestamp in seconds since \sphinxstyleemphasis{1970-01-01T00:00:00Z}.

Only present if \sphinxcode{\sphinxupquote{paramTimestamp}} is configured in \sphinxstyleemphasis{metadata} options.
\\
\hline
paramTimestamp
&
unsignedInt
&
Y
&
Y
&
31
&
Alert parameter timestamp, seconds since \sphinxstyleemphasis{1970-01-01T00:00:00Z}.

Only present if \sphinxcode{\sphinxupquote{paramTimestamp}} configured in \sphinxstyleemphasis{metadata} options.
\\
\hline
latitude
&
signedInt
&
Y
&
Y
&
24
&
Latitude in milliminutes, if present in \sphinxstyleemphasis{metadata} options.

Divide by 60000 to get decimal degrees (round to 6 decimal places precision).
\\
\hline
longitude
&
signedInt
&
Y
&
Y
&
25
&
Longitude in milliminutes, if present in \sphinxstyleemphasis{metadata} options.

Divide by 60000 to get decimal degrees (round to 6 decimal places precision).
\\
\hline
altitude
&
signedInt
&
Y
&
Y
&
20
&
Altitude in meters above sea level, if present in \sphinxstyleemphasis{metadata} options.
\\
\hline
\end{tabular}
\par
\sphinxattableend\end{savenotes}


\paragraph{paramAlertChange (MIN 103)}
\label{\detokenize{otaapi:paramalertchange-min-103}}\label{\detokenize{otaapi:paramalertchange}}
Sent automatically when an \sphinxstylestrong{alert} condition has changed from the last reported value.


\begin{savenotes}\sphinxattablestart
\centering
\sphinxcapstartof{table}
\sphinxcaption{paramAlertChange}\label{\detokenize{otaapi:id28}}
\sphinxaftercaption
\begin{tabular}[t]{|\X{25}{100}|\X{10}{100}|\X{5}{100}|\X{5}{100}|\X{5}{100}|\X{50}{100}|}
\hline
\sphinxstyletheadfamily 
Field Name
&\sphinxstyletheadfamily 
Type
&\sphinxstyletheadfamily 
Optional
&\sphinxstyletheadfamily 
Fixed
&\sphinxstyletheadfamily 
Size
&\sphinxstyletheadfamily 
Comments
\\
\hline
paramId
&
unsignedInt
&
N
&
Y
&
8
&
\sphinxstylestrong{Parameter} for which the \sphinxstylestrong{alert} has been raised.
\\
\hline
value
&
dynamic
\begin{itemize}
\item {} 
unsignedInt

\item {} 
signedInt

\item {} 
string

\item {} 
data

\end{itemize}
&
N
&
N/A
&
N/A
&
Value of the triggering parameter with data type corresponding to the parameter encoding:
\begin{itemize}
\item {} 
\sphinxcode{\sphinxupquote{uint8}}, \sphinxcode{\sphinxupquote{uint16}} use \sphinxstyleemphasis{unsignedInt}

\item {} 
\sphinxcode{\sphinxupquote{int8}}, \sphinxcode{\sphinxupquote{int16}}, \sphinxcode{\sphinxupquote{int32}} use \sphinxstyleemphasis{signedInt}

\item {} 
\sphinxcode{\sphinxupquote{float32}}, \sphinxcode{\sphinxupquote{string}}, \sphinxcode{\sphinxupquote{base64string}} use \sphinxstyleemphasis{string}

\item {} 
\sphinxcode{\sphinxupquote{raw}} uses \sphinxstyleemphasis{data}

\end{itemize}
\\
\hline
changeValue
&
dynamic
\begin{itemize}
\item {} 
unsignedInt

\item {} 
signedInt

\item {} 
string

\item {} 
data

\end{itemize}
&
N
&
N/A
&
N/A
&
Relative change from prior value of the triggering parameter with data type corresponding to the parameter encoding:
\begin{itemize}
\item {} 
\sphinxcode{\sphinxupquote{uint8}}, \sphinxcode{\sphinxupquote{uint16}} use \sphinxstyleemphasis{unsignedInt}

\item {} 
\sphinxcode{\sphinxupquote{int8}}, \sphinxcode{\sphinxupquote{int16}}, \sphinxcode{\sphinxupquote{int32}} use \sphinxstyleemphasis{signedInt}

\item {} 
\sphinxcode{\sphinxupquote{float32}}, \sphinxcode{\sphinxupquote{string}}, \sphinxcode{\sphinxupquote{base64string}} use \sphinxstyleemphasis{string}

\item {} 
\sphinxcode{\sphinxupquote{raw}} uses \sphinxstyleemphasis{data}

\end{itemize}
\\
\hline
timestamp
&
unsignedInt
&
Y
&
Y
&
31
&
Alert timestamp, seconds since \sphinxstyleemphasis{1970-01-01T00:00:00Z}.

Only present if configured in \sphinxstyleemphasis{metadata} options.
\\
\hline
parameters
&
array
&
Y
&
N
&
100
&
A set of parameter values defined by the \sphinxstylestrong{alert} configuration.
\\
\hline\begin{itemize}
\item {} 
paramId

\end{itemize}
&
unsignedInt
&
N
&
N/A
&
8
&
The \sphinxcode{\sphinxupquote{paramId}} being reported.
\\
\hline\begin{itemize}
\item {} 
value

\end{itemize}
&
dynamic
\begin{itemize}
\item {} 
unsignedInt

\item {} 
signedInt

\item {} 
string

\item {} 
data

\end{itemize}
&
N
&
N/A
&
N/A
&
Value with data type corresponding to the parameter encoding:
\begin{itemize}
\item {} 
\sphinxcode{\sphinxupquote{uint8}}, \sphinxcode{\sphinxupquote{uint16}} use \sphinxstyleemphasis{unsignedInt}

\item {} 
\sphinxcode{\sphinxupquote{int8}}, \sphinxcode{\sphinxupquote{int16}}, \sphinxcode{\sphinxupquote{int32}} use \sphinxstyleemphasis{signedInt}

\item {} 
\sphinxcode{\sphinxupquote{float32}}, \sphinxcode{\sphinxupquote{string}}, \sphinxcode{\sphinxupquote{base64string}} use \sphinxstyleemphasis{string}

\item {} 
\sphinxcode{\sphinxupquote{raw}} uses \sphinxstyleemphasis{data}

\end{itemize}
\\
\hline\begin{itemize}
\item {} 
paramTimestamp

\end{itemize}
&
unsignedInt
&
Y
&
N/A
&
31
&
Parameter timestamp in seconds since \sphinxstyleemphasis{1970-01-01T00:00:00Z}.

Only present if \sphinxcode{\sphinxupquote{paramTimestamp}} is configured in \sphinxstyleemphasis{metadata} options.
\\
\hline
paramTimestamp
&
unsignedInt
&
Y
&
Y
&
31
&
Alert parameter timestamp, seconds since \sphinxstyleemphasis{1970-01-01T00:00:00Z}.

Only present if \sphinxcode{\sphinxupquote{paramTimestamp}} configured in \sphinxstyleemphasis{metadata} options.
\\
\hline
latitude
&
signedInt
&
Y
&
Y
&
24
&
Latitude in milliminutes, if present in \sphinxstyleemphasis{metadata} options.

Divide by 60000 to get decimal degrees (round to 6 decimal places precision).
\\
\hline
longitude
&
signedInt
&
Y
&
Y
&
25
&
Longitude in milliminutes, if present in \sphinxstyleemphasis{metadata} options.

Divide by 60000 to get decimal degrees (round to 6 decimal places precision).
\\
\hline
altitude
&
signedInt
&
Y
&
Y
&
20
&
Altitude in meters above sea level, if present in \sphinxstyleemphasis{metadata} options.
\\
\hline
\end{tabular}
\par
\sphinxattableend\end{savenotes}


\paragraph{positionChangeAlert (MIN 104)}
\label{\detokenize{otaapi:positionchangealert-min-104}}\label{\detokenize{otaapi:positionchangealert}}
Sent automatically when movement has been detected and \sphinxstyleemphasis{geolocation} alerts are configured.


\begin{savenotes}\sphinxattablestart
\centering
\sphinxcapstartof{table}
\sphinxcaption{positionChangeAlert}\label{\detokenize{otaapi:id29}}
\sphinxaftercaption
\begin{tabular}[t]{|\X{25}{100}|\X{10}{100}|\X{5}{100}|\X{5}{100}|\X{5}{100}|\X{50}{100}|}
\hline
\sphinxstyletheadfamily 
Field Name
&\sphinxstyletheadfamily 
Type
&\sphinxstyletheadfamily 
Optional
&\sphinxstyletheadfamily 
Fixed
&\sphinxstyletheadfamily 
Size
&\sphinxstyletheadfamily 
Comments
\\
\hline
latitude
&
signedInt
&
N
&
Y
&
24
&
Latitude in milliminutes, if present in \sphinxstyleemphasis{metadata} options.

Divide by 60000 to get decimal degrees (round to 6 decimal places precision).
\\
\hline
longitude
&
signedInt
&
N
&
Y
&
25
&
Longitude in milliminutes, if present in \sphinxstyleemphasis{metadata} options.

Divide by 60000 to get decimal degrees (round to 6 decimal places precision).
\\
\hline
altitude
&
signedInt
&
N
&
Y
&
20
&
Altitude in meters above sea level, if present in \sphinxstyleemphasis{metadata} options.
\\
\hline
speed
&
unsignedInt
&
N
&
Y
&
8
&
Speed of travel in km/h.
\\
\hline
distance
&
unsignedInt
&
N
&
Y
&
31
&
Distance moved in tenths of meters, from previous location.
\\
\hline
timestamp
&
unsignedInt
&
Y
&
Y
&
31
&
Alert timestamp, seconds since \sphinxstyleemphasis{1970-01-01T00:00:00Z}.

Only present if configured in \sphinxstyleemphasis{metadata} options.
\\
\hline
\end{tabular}
\par
\sphinxattableend\end{savenotes}


\paragraph{error (MIN 200)}
\label{\detokenize{otaapi:error-min-200}}\label{\detokenize{otaapi:error}}
Sent automatically when a Modbus communication or configuration problem is detected.


\begin{savenotes}\sphinxattablestart
\centering
\sphinxcapstartof{table}
\sphinxcaption{error}\label{\detokenize{otaapi:id30}}
\sphinxaftercaption
\begin{tabular}[t]{|\X{25}{100}|\X{10}{100}|\X{5}{100}|\X{5}{100}|\X{5}{100}|\X{50}{100}|}
\hline
\sphinxstyletheadfamily 
Field Name
&\sphinxstyletheadfamily 
Type
&\sphinxstyletheadfamily 
Optional
&\sphinxstyletheadfamily 
Fixed
&\sphinxstyletheadfamily 
Size
&\sphinxstyletheadfamily 
Comments
\\
\hline
err
&
unsignedInt
&
N
&
Y
&
8
&
\begin{DUlineblock}{0em}
\item[] Error code.
\item[] 0 - OK / Device communication recovered
\item[] 1 - Device communication lost
\item[] 2 - ReportingId not found in configuration
\item[] 3 - AlertId not found in configuration
\end{DUlineblock}
\\
\hline
errText
&
string
&
N
&
N
&
100
&
Verbose error text corresponding to the error code.
\\
\hline
timestamp
&
unsignedInt
&
Y
&
Y
&
31
&
Error timestamp, seconds since \sphinxstyleemphasis{1970-01-01T00:00:00Z}.

Only present if configured in \sphinxstyleemphasis{metadata} options.
\\
\hline
\end{tabular}
\par
\sphinxattableend\end{savenotes}


\section{LSF API Integration}
\label{\detokenize{lsfapi:lsf-api-integration}}\label{\detokenize{lsfapi::doc}}
\begin{sphinxadmonition}{warning}{Warning:}
This section is under construction!
\end{sphinxadmonition}


\subsection{Properties}
\label{\detokenize{lsfapi:properties}}
See {\hyperref[\detokenize{configuration:config-properties}]{\sphinxcrossref{\DUrole{std,std-ref}{Configuration}}}}


\subsection{Shell Commands}
\label{\detokenize{lsfapi:shell-commands}}\begin{itemize}
\item {} 
\sphinxcode{\sphinxupquote{modbus config set}}

\item {} 
\sphinxcode{\sphinxupquote{modbus config get}}

\item {} 
\sphinxcode{\sphinxupquote{modbus config del}}

\item {} 
\sphinxcode{\sphinxupquote{modbus reporting set}}

\item {} 
\sphinxcode{\sphinxupquote{modbus reporting get}}

\item {} 
\sphinxcode{\sphinxupquote{modbus reporting del}}

\item {} 
\sphinxcode{\sphinxupquote{modbus analytics set}}

\item {} 
\sphinxcode{\sphinxupquote{modbus analytics get}}

\item {} 
\sphinxcode{\sphinxupquote{modbus analytics del}}

\end{itemize}


\subsection{Events}
\label{\detokenize{lsfapi:events}}
None.


\subsection{Functions}
\label{\detokenize{lsfapi:functions}}
None.



\renewcommand{\indexname}{Index}
\printindex
\end{document}